\documentclass[12pt]{paper}

\usepackage[margin=1in]{geometry}
\usepackage{tikz}
\usepackage{Schwieg}
\title{Problem Set 4.2}
\author{Samuel Barker\and Timothy Schwieg\and Rafeh Qureshi\and Daniel Noriega}
\begin{document}
\maketitle


\section*{Setup}
We set up a model\footnote{We revisited Becker's \textit{The Economic
    Approach to Human Behavior} to confirm our thoughts} in terms of
the differences to utility crime inflicts upon the perpetrator and the
victim. Looking at the level of crime in society (A), we have the
utility gain by the perpetrator (1 below) and the loss of gain to the
victim (2 below). We take it that crime is socially sub-optimal
(assuming we weight everyone equally) if for a given level of criminal
activity (A), (3) holds 


\begin{align}
\frac{\partial U_p}{\partial A} & > 0 \\
\frac{\partial U_v}{\partial A} & < 0 \\
 \frac{\partial U_p}{\partial A} - \frac{\partial U_v}{\partial A} &> 0
\end{align}
Aggregating to the group level, we can sum up over the utility gains
to the criminals in society and the utility losses to the victims to
get analogues of these at the societal level:
\begin{align}
B(A) & > 0 \\
H(A) & < 0 \\
B(A)- H(A) \equiv F(A) &> 0
\end{align}
where $B(A)$,$H(A)$ and $F(A)$ represent the utility gains to the
criminals, utility losses to the victims and the aggregate societal
loss respectively.
\subsection*{Policing}
Society spends a given amount of resources on policing to increase the
probability of conviction $p$. Clearly, it is not optimal for society
to expend all its resources to getting $p$ to as high as possible, i.e
if we have the marginal benefit of spending another dollar by the
government to society as a constant $b_0$ (which represents the
opportunity cost of government expenditure), society spends on
policing until the marginal benefit to society of another dollar spent
on policing is equal to $b_0$. We will analyze this further later, but
for now we can say the optimal level for policing, due to this
criterion, is $\hat p$. Because policing is fully captured in $p$, we
can just look at the costs of policing and analyze changes in societal
loss F(A) with respect to p. Thus, we have the function of policing
costs as $C(p,A)$. Note, that the level of criminal activity itself is
a function of $p$.

\section*{a}
We think of this problem as a cost minimization problem, given the
aggregate level of ``utility'' in a society is fixed at $U_{opt}$
\begin{align}
 \min_{f,m} V(A,p,f) = \int_0^a H(a) - B(a) + C(p,a) da \\
  H(A_{opt}) - B(A_{opt}) + C(p,A_{opt}) = 0
\end{align}
Note the criteria in (8) describes the efficient amount of criminal
activity in society by the social planner. Here, the costs of policing
($C$) are offset by the benefit to the society($F(A)$). We look at the
mechanisms through which the social planner decreases $A$ to socially
optimal levels:


\begin{align*} 
 \frac{\partial V}{\partial p} & = \frac{\partial C}{\partial p} + \frac{\partial{H(A)}}{ \partial{A}} \frac{ \partial{A}}{\partial{p}} - \frac{\partial{B(A)}}{\partial{A}}\frac{\partial{A}}{\partial{p}} + \frac{\partial{C}}{\partial{A}}\frac{\partial{A}}{\partial{p}} &=0\\
  \frac{\partial V}{\partial f} & = \frac{\partial{H(A)}}{\partial{A}} \frac{\partial{A}}{\partial{f}} - \frac{\partial{B(A)}}{\partial{A}}\frac{\partial{A}}{\partial{f}} + \frac{\partial{C}}{\partial{A}}\frac{\partial{A}}{\partial{f}} &= 0
\end{align*}


Intuitively, with respect to a change in probability of getting
caught, we would see the costs of policing increase directly with
respect to the change in $p$ and decrease indirectly through the
change in activity (A) if policing costs are increasing with criminal
activity ($\frac{\partial{C}}{\partial{A}}>0$) and criminal activity is decreasing
with probability of conviction ($\frac{\partial{A}}{\partial{p}}<0$). Similarly, we
would see the benefit to the criminals decrease with an increase in
$p$ through the criminal activity decreasing with $p$. The loss to
victims similarly decreases through the $\frac{\partial{A}}{p}$. \par

Let's say the level of policing is fixed at $\hat{p}$. Looking at the
change in societal costs with the fine, we see that changing $f$
increases the costs to society through the decrease in benefit of the
criminals (due to it bringing about a decrease in criminal activity -
$\frac{\partial{A}}{\partial{f}}<0$). It also decreases the costs to society through
it decreasing the loss of victims. \par
With respect to fines, we can recast the problem to our familiar
optimality conditions through voluntary trade. Namely, the pecuniary
costs of the fine must be analogous to the price in the sense that it
compensates the victims for their dis-utility. More formally, the
optimal criminal activity will be such that the marginal loss to
society is zero (F(A)=0), but then,
$$ H(A_{opt}) + C(p,A_{opt}) = B(A_{opt}) = p f$$. Namely, the fine
will be such that the expected value of the fine (probability of
getting caught $p$ times the fine paid when caught $f$) will
compensate the policing costs of bringing about that $p$ and the harm
to victims from the criminal activity $A$. Thus, it is similar to the
price levied on victims (in expectation) for them to purchase the
level of criminal activity $A$.  \pagebreak
\section*{b}
Certainly, to offset the harm to the victims, fines are recommended
form of punishment, but if the criminals don't have enough assets to
pay, we must concentrate on exerting another cost to them such as
through imprisonment (or torture, guillotine etc). \par
Here, however, there are two additional costs; in addition to the
policing costs $C$, we will have the costs of the punishment
itself. We represent this cost as $pZ(A,z)$, where $z$ is the
amount of punishment brought about (i.e z could be days of prison
sentence) and $p$ shows up in expectation as we look at the cost of
punishment for the individuals that as caught
($\mathbbm{E}(Z(A,z)) = pZ(A,z) $). The condition for optimality here
is that the level of harm exerted on to the criminals must equal the
reduction in the harm exerted on to society. Namely, we shall have:
\begin{align*} 
 \frac{\partial V}{\partial p} & = \frac{\partial C}{\partial p} + Z(A,z) + \frac{\partial Z(A,z)}{\partial A}\frac{\partial A}{\partial p} + \frac{\partial{H(A)}}{ \partial{A}} \frac{ \partial{A}}{\partial{p}} - \frac{\partial{B(A)}}{\partial{A}}\frac{\partial{A}}{\partial{p}} + \frac{\partial{C}}{\partial{A}}\frac{\partial{A}}{\partial{p}} &=0\\
  \frac{\partial V}{\partial z} & = \frac{\partial{H(A)}}{\partial{A}} \frac{\partial{A}}{\partial{z}} + \frac{\partial{Z}}{\partial{A}} \frac{\partial{A}}{\partial{z}} + \frac{\partial{Z}}{\partial{z}} - \frac{\partial{B(A)}}{\partial{A}}\frac{\partial{A}}{\partial{f}} + \frac{\partial{C}}{\partial{A}}\frac{\partial{A}}{\partial{f}} &= 0
\end{align*}
Above, we see that we would like the reduction in the societal loss
($F(A) + C(p,A)$) with respect to $p$ and $f$ respectively both to
equal the change in societal loss due to the punishment. Namely, the
change in harm brought about by the changes in $p$ and $f$
respectively must be offset by the marginal benefit brought on to
society by the changes in $p$ or $f$. \par
The crimes that will occur in equilibrium will be those where the
individuals have a small enough probability of getting caught or
individuals value the risk of getting caught low enough that their
benefit of one more unit of crime exceeds (or equals) their expected
cost. Here, we could expect that the individuals who commit crime be
relatively risk-loving (they are essentially gambling with respect to
the probability of getting away with their crime and utility benefit
$B(A)$. Otherwise, we could also have the criminals commit crime till
they are indifferent between their expected dis-utility from the
punishment. One special case of this is when the criminals have very
low aversion to the punishment $Z$ and may actually may derive some
benefit from it. i.e This would happen if they are homeless and would
be okay with imprisonment; then, they are likely to commit a crime.
\pagebreak
\section*{c}
The delay between the commission of crime the time would be captured
in the societal loss from the criminal activity. Namely, there will be
an increase in the social cost due to the interest accrued in the time
between the committing of the crime and the fine that is accrued. Take
the optimal fine in (a) of
$$ f = 1/p \times (H(A_{opt}) + C(p,A_{opt})) $$ Then, due to the time
between the crime and the conviction, the fine must
be aggrandized to compensate the individual for what the monetary
equivalent of the dis-utility would have accrued in that time:
$$ f = (1+r)^t \times (H(A_{opt}) + C(p,A_{opt}))/p $$ Similarly, we must
be careful in (b) to incorporating the gain in the societal loss over
time
\begin{align*} 
   Z(A,z) + \frac{\partial Z(A,z)}{\partial A}\frac{\partial A}{\partial p} &= (1+r)^t *(\frac{\partial C}{\partial p}  + \frac{\partial{H(A)}}{ \partial{A}} \frac{ \partial{A}}{\partial{p}} -  + \frac{\partial{C}}{\partial{A}}\frac{\partial{A}}{\partial{p}})\\
 \frac{\partial{Z}}{\partial{A}} \frac{\partial{A}}{\partial{z}} + \frac{\partial{Z}}{\partial{z}} &= (1+r)^t*(\frac{\partial{H(A)}}{\partial{A}} \frac{\partial{A}}{\partial{z}} - \frac{\partial{B(A)}}{\partial{A}}\frac{\partial{A}}{\partial{f}} + \frac{\partial{C}}{\partial{A}}\frac{\partial{A}}{\partial{f}})
\end{align*}
Thus, here the optimal level of $z$ (i.e the level of punishment) will
increase to diminish crime further in order to compensate for the
increase in the societal harm brought on by the delay.
\section*{d}
Yes; note that the fine compensates the differential in the marginal
loss to society (i.e the marginal societal loss in summing the victim
and perpetrators' utility and the cost of enforcing). It is
independent in the optimal setup of the wealth of the
perpetrator. Recall, again, the analogy to the criminal activity (A)
being what the criminal buys for the price of the fine $f$. Then, of
course, the price will be independent of the wealth of the criminal
and will instead depend on the dis-utility of the victim and the costs
of transaction (namely, the C(p,A)).\\ 
Note that with the example of (b) where the person does not have
enough resources to pay out the harm of his actions (which would be
the case for murder), the richer person might end up experiencing more
punishment as his time is worth more. Here, however, the richer person
is still expected to levy his resources to avoiding the jail time
(through legal fees etc).  \pagebreak
\section*{e}
If the punishment is through fines, that system might not be
considering the cost of enforcing the equilibrium level of policing
(getting $p$) in society. Similarly, if the punishment is through
additional means, this may be overlooking the costs to society of the
actual punishment. \par 
If, however, the costs to society for the imprisonment and the
policing are nil, the system could indeed be optimal if it met the
condition in (8). Do note that the expected fine should take into
account the probability that the victim would not be convicted: $$ f =
1/p \times (B(A_{opt})  + C(p,A_{opt})) $$ Also, the necessary and
sufficient condition is for this to hold at the margin. Thus, we need
the marginal level of crime to be such that the victim is compensated
in expectation by the criminal.   
\section*{f}
We can view the added benefit to the perpetrator as a social
externality. Then, the benefit to the perpetrators (and thus society)
goes up by the level of crime. Thus, our previous level of societal
harm gets a benefit equal to the aggregate benefit to criminals of the
additional `utility' criminals experience through others' criminal
activity. Say that $\epsilon(A)$ is the added benefit of others' criminal
activity to the criminals. Then, we can define $\hat H(A)$ such that  
 $$H(A) - B(A) + \epsilon(A) = \hat H(A) - B(A)$$ 
 Here, the analysis is similar to before, with a greater benefit to
 criminals in society, which would mean that there would be higher
 levels of crime in equilibrium, and also possibly higher fines to
 compensate the victims for the added benefit of the added crime (i.e
 $B'(A) = H'(A) + \epsilon '(A) = f)$.
 
\end{document}
