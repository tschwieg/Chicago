\documentclass[12pt]{paper}

\usepackage{fullpage} % Package to use full page
\usepackage{parskip} % Package to tweak paragraph skipping
\usepackage{tikz} % Package for drawing
\usepackage{amsmath}
\usepackage{hyperref}
\usepackage[margin=1in]{geometry}

\title{Problem Set 7: Question 2}
\author{Samuel Barker, Daniel Noriega, Rafeh Qureshi, Timothy Schwieg}


\begin{document}
\maketitle


\section*{Model Setup}

We model the environment as imperfect competition between networks (like TWC) and
distributors (like AT\&T). Each firm faces a downward sloping
demand curve for their differentiated product. We do not imply that
there is any sense of ``market power'' that each firm has, only that
they have the ability to mark up their prices, which allows for them
to bear fixed costs.


Distributors are imperfect substitutes for each other and they may not
all share the same price. Each firm will face their own firm-specific
demand, and will have some mark-up in price above where marginal cost
intersects their marginal revenue curves. Firms could earn profits in
these markets. However, no firm has monopoly power in these markets,
so the demand that they each face is not the total market demand. It
is just the amount of people that will buy their good at a particular
price, given the prices of all their competitors.


\section*{Part a}

Q: Given that networks have zero marginal cost of
adding viewers, are they making a mistake to charge
distributors on a per-subscriber basis? Should
distributors be compensating networks in some other way?
\\

It is important that we make a proper distinction in the goods that
are transacted in each market. Networks produce channels as
their good, and sell access to this good to the distributors. Networks face fixed and marginal costs for
production of these channels. Distributors buy access to channels, and then sell network packages to
consumers. For simplicity's sake, we will consider that each
distributor has only one package that contains all the channels that
it pays for access to.


Assume that all the consumers are homogeneous, and have some
willingness to pay for access to the networks content. Since there are
many networks, demand for the distributor's content will still be
downward sloping. The value of a distributor carrying the network's
content is the customers` willingness to pay multiplied by the number
of customers that are purchasing the good. Call this willingness to
pay $v$, and the number of customers $q$. The value that a distributor
receives for carrying a network is $qv$, which is therefore the
distibrutors` willingness to pay for that content.


The first alternative for the pricing structure of the market would be
that all distributors faces a single price for access to the
channels. Each firm that wanted a network's content would pay $C$
dollars. Firms with $qv > C$ would buy this access, and receive a
consumer surplus of $qv - C$. Some firms would earn significant
surplus, while others would not acquire the network while still being
willing to pay for on a per-customer basis.


However, networks could also observe the $v$ for consumers and would
be able to perfectly price discriminate, absorbing consumer surplus
into producer surplus. By charging $t=v$ per customer, each
distributor is indifferent between carrying the network and not;
assume that they all do.


If a per-customer charge is implemented, all distributors buy the
network, and the consumer surplus is captured by the network. This
must be optimal from the network's perspective.

\section*{Part b}
Q: The merger would increase the retail price paid
by the households that subscribe to AT\&T's
competitors
\\

\textbf{Uncertain}. Before the merger, TW networks was a perfect price
discriminator. After the merger, they can do no worse than before, and
since competitors were indifferent before they
can do no better. The cost that each distributor pays to acquire TW
networks content will remain fixed at $t=v$.


Competitors with $q$ buyers were willing to pay $qv$ to have Time
Warner Networks` content, and when faced with a price of $qt=qv$
are indifferent between having TW networks or not. The merged
company cannot charge a higher price, as the competitor would leave
the market, and the revenue earned would then be zero.


The merged company would not choose to reduce the price either, as
they still have perfect information about the valuations of consumers,
and therefore the willingness to pay of the distributors. With this
information, perfect price discrimination is the optimal choice for
the profit maximizing firm.


Therefore the cost of obtaining TW networks has not changed for any of
AT\&T's competitors. The only source of changes in the price could
come from changes in the firm-specific demand of the competitors. This
firm-specific demand is dependent on the price of competitors,
particularly of AT\&T. From part (d) we infer that the price of AT\&T
could decrease. If the price of AT\&T decreased, then we would see the
firm-specific demand for its competitors decrease. This means the
price will decrease, and the quantity sold will decrease as well.


If AT\&T elects not to sell Time Warner content to its competitors,
then the marginal cost of the competitors will decrease, as they no
longer pay $t$ per customer, but their individual demand decreases, as
they no longer carry Time Warner content and could have fewer
customers. Their price could increase or decrease, and all that is
known is that the firm is indifferent, so its producer surplus remains
unchanged. Since demand could change as well, however, this does not
allow us to make predictions about the price.

% \textbf{Uncertain}. AT\&T's competitors will face higher costs of
% obtaining TWC products, but will have less market power. The overall
% effect of this could either be a reduction in price or an increase in
% price.

% Before the merger, Time Warner earned $t$ dollars per viewer on the
% channels that it sold to. There was no benefit to not selling it to
% any other competitor, as they faced nearly zero cost of selling their
% channels to distributors.

% After the merger occurs, The conglomerate faces an extra cost of
% selling their channel to other distributors. Since they are a
% distributor, when other distributors carry their channels, consumers
% have more substitutability available. This decreases the benefit of
% having AT\&T as a distributor, and makes consumers less willing to pay
% for AT\&T. As a result they face a cost of selling Time Warner channels
% to their competitors. Prior to the merger, they did not
% face any costs of selling their channels to other distributors. The
% costs of providing the channels have increased, so the benefit, the
% price, must increase as well. This means that the price $t$, that is
% charged to the competitors will increase.

% These firms now face higher costs of reaching out to their consumers,
% and must now re-evaluate their markups in prices. For higher costs,
% the same markup that these firms are using will discourage the very
% price-sensitive customers away. To offset this, the markups that the
% firms charge will be decreased. Higher costs effectively reduce the
% power that these firms have in the market, and lowers their ability to
% mark up prices, as they must stay ``competitive.'' The decrease in the
% mark-up could lead to a lower or higher price, depending on the
% relative magnitudes of each change.

% AT\&T definitely increases its power in the market, as it now has the
% ability to control the price at which its competitors receive TW
% networks. So the relative power of its competitors has decreased, as
% they have no control over the prices that AT\&T pays for any of its
% content.

\section*{Part c}

Q: If AT\&T acquired TW, then the marginal cost t would be zero
\\

\textbf{True}. The good that AT\&T provides is network access to
consumers. AT\&T faces no extra costs of providing TW access to a new
consumer once it has merged.


If AT\&T acquires Time Warner, they still face the
costs of producing the channels. One can imagine this as the
Time-Warner portion of the merged company producing the channel, and
then selling this to the AT\&T portion at cost. The cost of AT\&T
carrying TW networks now is simply the cost of TW networks producing
content.


It is not reasonable to consider the TW-portion selling it to AT\&T at
a price $t'$ per consumer that AT\&T sells to, as TW content will also
be sold to other networks. This is therefore a fixed cost that AT\&T
bears, distributes and sells to other distributors as well.


This changes the pricing schema that AT\&T faces, as instead of being
charged per consumer, they are charged per channel that they
produce. Their marginal cost of producing more channels is non-zero,
and presumably quite high. But the good that AT\&T is producing is
access to networks for its consumers. On this front, the only costs of
adding a new consumer are the marginal costs $c$. AT\&T paid on the
extensive margin to obtain access to TW networks at the cost of
producing it, and therefore does not face a marginal cost of adding
extra customers. Note that this zero marginal cost only occurs on the
dimension of adding extra customers. Since that is the good that AT\&T
is selling, this is the dimension we consider.


Since the good that networks produce is channels, the costs
that AT\&T faces are the costs of producing that channel. These are
presumably quite high but have become fixed costs for the firm. Once
they have chosen on the external margin to participate in providing TW
networks, they bear this cost.

\section*{Part d}

Q: The merger would lower the retail price paid by
  households  that had been subscribing to AT\&T.
  Compare the magnitude of the gain or loss to the
  gain or loss that you calculated in part (b)
\\

\textbf{Uncertain}. The marginal cost of providing AT\&T to consumers
has decreased. However a case remains where AT\&T has the option not
to distribute TW network content, and charge a ``premium'' that could
increase the price as a result.


If the Time-Warner content is sold to other distributors, the
firm-specific demand that AT\&T faces has not changed at all. This
means that the marginal revenue that AT\&T earns per customer has not
changed either.


AT\&T will choose to produce where marginal cost equals marginal
revenue, and this must occur at a greater quantity and lower price
than they were selling at before the merger. In this case the retail
price paid by households has decreased, and consumers are better off.

Consumers gain here, and the competitors of AT\&T in part (b) now have
a reduced market share because of the lower price of AT\&T. This means
that their demand curve has moved inwards, and their price and
quantity sold have decreased. Consumers in both markets are better
off, so the merger has a net beneficial effect for consumers. It
cannot be that the quantity of consumers decreases, as the decreased
demand leading to lower consumers in the competitors is caused by a
lower price in the AT\&T market. These customers are just switching to
the cheaper alternative.


The other option remains for AT\&T not to sell Time-Warner content to
its competitors. Then AT\&T has monopoly power over Time-Warner
content, and increases its market power. This means that the demand
curve shifts outwards, and it is possible that the prices paid by
consumers increase. AT\&T is less substitutable to the other
distributors as it is the only distributor that carries Time-Warner
networks, and this could translate into market power.

In this world, the marginal costs for its competitors have decreased,
since they are no longer paying for Time-Warner Cable, but their
demand has decreased as well. Nothing can be said about whether the
price has increased or decreased however, so we cannot compare the
magnitude of gains and losses in both markets.


The question of whether or not AT\&T decides to sell access to the TW
networks is whether or not the benefit of exclusively having Time
Warner content would be higher than selling the content to the
competitors. The revenue earned by selling to competitors is given by
$\sum q_i v$ where $q_i$ is the quantity of people buying from the
competitors. If the increase in revenue from the increased market
power is less than this, then they will sell to their competitors, and
otherwise will maintain exclusivity. The revenue of selling TW`s
content to competitors is most likely to be relatively small (compared
to exclusivity proceedings) when AT\&T already has a large market
share, and more likely to be large when AT\&T has a small market
share, but the exact numbers depend on the substitutability of Time
Warner networks.

% However AT\&T has also increased its market power by way
% of the merger.  By owning one of the networks that other distributors
% need access to, they have increased their power in the market. This
% means that they are capable of choosing a higher mark-up for their
% price. The lower costs are a force that is driving the price down, but
% the increased market power drives the price upwards. The overall price
% change cannot be determined without more information.



% Where is the source of the market power that they have earned by way
% of the merger? Since they can charge a higher price to obtain TW
% networks to all of their consumers, their marginal costs have
% increased. The amount that this cost increases is dictated by AT\&T,
% and is the source of their power. As they raise this cost, each of its
% competitors will have to increase their price, and produce less
% quantity. The relative power of AT\&T in the market will increase with
% the merger.

% TODO: We still need to compare the magnitudes of the gains and losses
% with part (b).

\section*{Part e}
Q: To the extent that the merger results in a shift
  of consumers toward AT\&T, that is inefficient


\textbf{Uncertain}. If AT\&T could increase the price it charges to
customers at the same time that it increases its customer-base (we
assume that the market size has not changed), this would imply that
the merger is inefficient as the dead-weight would inevitably
increase.


If we were in the world where AT\&T reduces its price and continues to
sell to its competitors, then the effect of the shift would depend on
how the mark-up is changing. In particular, it would depend on the
relation between the elasticity of demand, the change of the marginal
revenue curve, and the actual change of the marginal cost curve. It
could be the case that AT\&T increased the mark-up. AT\&T has reduced
its cost, but if the mark-up increased, the dead-weight loss would
increase, which would imply that the allocation is more inefficient.


In the case where AT\&T Monopolizes Time Warner networks content, it
has increased its monopoly power, and decreased its marginal
costs. This means that the dead weight loss will increase, and this
outcome is more inefficient as well.

Inefficiency in the market is likely to increase in both outcomes, in
the sense that the dead weight loss in the distributor market has
increased.


Nonetheless, consumers could face overall lower costs thereby
increasing consumer surplus, at the same time that producer surplus
increases. Efficiency could be understood as production at lowest
cost, and even in the case that AT\&T mark-ups increased, if the final
price that consumers face is lower than the initial price, both
consumer and producer surpluses may increase at the same time.


\section*{Part f}
Q: Now put aside the ATT-TW merger and think about mergers
generally. The merging parties always argue that federal authorities
should permit the merger because it will reduce costs (e.g., avoiding
duplication) per unit quality and those costs savings are passed onto
consumers. True, False, or Uncertain: The mergers should be blocked
anyway because any pair of two potential merging partners could
realize the cost savings through contracting rather than through
merger.


\textbf{False}. Although contracts may seem like a feasible
alternative in principle, a merger would capture future possibilities
that would be very hard (and possibly very costly) to specify in a
contract. Therefore, even if specifying a contract that perfectly
substitutes a merger were possible, it would be a more expensive
alternative, and therefore, it would not be possible to achieve the
same cost savings that a merger would pose.


Consider the number of scenarios that could potentially occur in the
future, which for practical purposes can be thought of as infinite
(very large in any case). A contract stipulating enough conditions to
accurately mimic a merger would have to specify terms that capture an
essentially infinite set of scenarios. One could only imagine how
costly drafting and enforcing such a contract would be; costs that
could decrease potential savings which could be realized through a
merger.

\section*{Part g}
Q: Would your answer to (f) be different if the data showed a positive
causal effect of many (although not all) mergers on industry output?


No, it would not. If such an effect was apparent, although it could be
a case of market power (which the question seem to suggest), it could
also be the case that in fact cost savings are achieved, increasing
efficiency and allowing consumers to increase consumption. The
evidence that there is a causal relation between mergers and an
increase in industry output could also support the idea that mergers
may be the most efficient option for different parties to interact in
production relations, as opposed to contractual alternatives.

\end{document}