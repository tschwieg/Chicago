\documentclass[12pt]{paper}

\usepackage[margin=1in]{geometry}
\usepackage{amsmath}
\usepackage{bm}
\usepackage{amsthm}
\usepackage{mathtools}
\usepackage{bbm}


\DeclareMathOperator{\diam}{diam}
\DeclareMathOperator{\interior}{int}
\DeclareMathOperator{\close}{cl}
\DeclareMathOperator*{\argmin}{arg\,min}
\DeclareMathOperator*{\argmax}{arg\,max}

\newcommand{\met}[1]{d \left ( #1 \right )}
\newcommand{\brak}[1]{ \left [ #1 \right ] }
\newcommand{\cbrak}[1]{ \left \{ #1 \right \}}
\renewcommand{\vec}[1]{ \bm{ #1 }}
\newcommand{\abs}[1]{\left \lvert #1 \right \rvert}
\newcommand{\seq}[1]{{\left \{ #1 \right \}}}
\newcommand{\conj}[1]{ \overline{ #1 } }
%\newcommand{\close}[1]{ \bar{ #1 } }
\newcommand{\set}[1]{\left \{ #1 \right \}}
\newcommand{\Lim}{\lim\limits}
\newcommand{\compose}{\circ}
\newcommand{\inv}[1]{{#1}^{-1}}
\newcommand{\compl}[1]{{#1}^{c}}



\newcommand{\setR}{ \mathbb{R} }
\newcommand{\setQ}{ \mathbb{Q} }
\newcommand{\setZ}{ \mathbb{Z} }
\newcommand{\setN}{ \mathbb{N} }

\newcommand{\plim}{ \overset{p}{\to} }
\newcommand{\mean}[2][N]{ \overline{ #2 }_{#1}}
\newcommand{\exV}[1]{\mathbb{E} \left [ #1 \right ]}
\newcommand{\Vari}[1]{\mathbb{V} \left ( #1 \right )}
\newcommand{\est}[2][N]{ \widehat{ #2 }_{#1}}
\newcommand{\indicate}[1]{ \mathbbm{1}_{\{#1\}}}
\newcommand{\convDist}{ \overset{d}{\to}}
\newcommand{\unif}{\emph{U}}
\newcommand{\normal}{\mathcal{N}}

\newcommand{\deriv}[2]{\frac{ \partial #1}{ \partial #2}}

\DeclarePairedDelimiter{\ceil}{\lceil}{\rceil}
\DeclarePairedDelimiter{\floor}{\lfloor}{\rfloor}
\DeclarePairedDelimiter{\norm}{\lVert}{\rVert}



\title{Problem Set 6 - Question 2}
\author{ Timothy Schwieg \and Daniel Noriega \and Samuel Barker \and
  Rafeh Qureshi}

\begin{document}

\maketitle

\section*{What can you say about the relationship between prevalence
  and new infections? Separate your answer into a mechanical
  component, related to random pairings of partners and a behavioral
  component.}

We shall consider a model of the dynamics in which there are three
types of actors.  Actors who do not have syphilis, actors who have
contracted syphilis but are not yet contagious, and actors who have
syphilis and are still contagious. Any individual actor knows which
state he is in, but not the state of whomever he is pairing
with. All individuals are aware of the amount of contagious
individuals. All actors are aware of $y_t$. Let us normalize the
number of people in the world to $1$. In this way, if all people
sought the same amount of encounters, the probability of obtaining
syphilis would be given by $p = \frac{y_t}{N} = y_t$.

Every individual therefore has a state, $1$ for healthy,
$i \in \{ 2,3,... x, x+1 \}$ for individuals with the disease that are
not yet contagious, and $x+2$ for contagious. All information about
the individual is summarized by these states. Individuals in state $1$
move to state $2$ if they contract the disease, and otherwise remain
in state $1$. All individuals in state $2$ through $x+1$ move to the
next state with probability $1$. Contagious individuals in state $x+2$
remain there with probability $1-r$ and move to the healthy state with
probability $r$.

All actors in the model receive the same benefit from a sexual
encounter, and individuals only differ between the costs they face for
this encounter. Each individual does face a constant cost per
encounter taken as a monetary amount. This can be rationalized either
in time spent in seduction, or in actual costs spent on dates. This is
just a fixed cost among dates, and ensures that individuals that bear
no risk of obtaining the disease would not desire infinite
hook-ups. Healthy individuals face some cost $\beta$ if they contract the
disease, which is the time discounted present cost of obtaining the
disease. This $\beta$ accounts for all the utility and costs associated
with having the disease, in expectation. 
 
This cost function is state dependent, as individuals that have
syphilis cannot obtain the disease a second time. These individuals
still face the costs of an encounter, but not the disease risk. Let us
believe that there is some guilt factor associated with giving
syphilis to another person. Call this level $\gamma$. This guilt is only
felt if syphilis is translated to another person. In expectation, this
is $\alpha ( 1 - p)\gamma$.  Non-contagious individuals that have the disease do
not bear the guilt, while contagious individuals do. This means that
non-contagious diseased actors have the least costs to
copulation. 

The benefit of $\alpha$ sexual encounters is given by $u(\alpha)$ which is an
increasing and concave function. Individuals in equilibrium will
copulate to the point where marginal benefit equals marginal
cost. Note that the benefit is the same for every individual
regardless of state. All that differs between individuals is the
marginal cost.

% What is interesting is that in the dynamics of this system, the
% choices of the infected individuals are irrelevant to the dynamics of
% who gets sick. A healthy person's does not observe the utility
% functions of all other people, and has no idea how often sick people
% are copulating. He only observes $y_t$ and from that calculates his
% chance of obtaining the disease from a random encounter. That
% information is not based on the decisions made by sick individuals, as
% once they contract the disease, they become a statistic in $x$ time
% periods. Their behavior cannot affect this probability.

% The guilt parameters do affect the choices of the number of encounters
% for the contagious individuals. However there is no way for a person
% to screen for the disease ahead of time. this means the only
% information he has for the probability of contracting the disease is
% the percentage of contagious people.

The healthy individuals' choice does not depend on what the sick
people are doing. They have no knowledge of the choices made by the
sick parties, and only observe the quantity of infected
individuals. They do not even discover the costs of their actions
until $x$ periods into the future. 

When they decide to engage in the sexual marketplace, they
face a different probability of contracting the disease. The
contagious, and not yet contagious sick individuals will choose
different levels of copulation in general than the healthy
individuals. The probability of them actually obtaining the disease is
then a mixture of each groups' choice of copulation. Let us call this
endogenous probability $P^{*}$

There are $y_t$ contagious people, who choice to fornicate $a_c$
times. The healthy individuals fornicate $\alpha$ times, and the
non-contagious but infect individuals fornicate $a_n$ times.  These
choices are given by:
\begin{align*}
  \alpha &= \argmax_{\alpha} \quad u(\alpha) - \delta\alpha - \beta*( 1 - (1-p)^{\alpha})\\
  a_c &= \argmax_{a_c} \quad u( a_c) - \delta a_c - \alpha ( 1 - p)\gamma\\
  a_n &= \argmax_{a_n} \quad u( a_n) - \delta a_n                   
\end{align*}

The probability that a healthy individual contracts the disease from a
single encounter is given by


\begin{equation*}
  P^{*} = \frac{a_c y_t}{a_c y_t + \alpha h_t + a_n (1 - y_t - h_t)}
\end{equation*}

where $h_t$ is the number of healthy individuals at the time $t$. This
is not simply the number of contagious people, as we see each group of
people make different choices about the amount of people they choose
to sleep with.

We can see that the choices of the sick individuals is only present in
the level of $P^{*}$, as the dynamics of sick people are completely
irrelevant of their choice of fornication. People who are not
contagious will become contagious with probability one, and the
contagious individuals become healthy with probability $r$. Neither
are endogenous to their choices, and they can only affect the dynamics
of healthy individuals.

For $x = 3$, the transition probability matrix for this system is
given below:

\begin{equation*}
\begin{pmatrix}
  (1-P^{*})^{\alpha^{*}} & 1 - (1-P^{*})^{\alpha^{*}} & 0 &0 &0\\
  0 & 0 & 1 & 0 & 0\\
  0 & 0 & 0 & 1 & 0\\
  0 & 0 & 0 & 0 & 1\\
  r & 0 & 0 & 0 & 1 -r
\end{pmatrix}
\end{equation*}

Where $P^{*}$ is given as above, and $\alpha^{*}$ is the choice for only
the healthy individuals. 


% As we can see, this depends only on the choices of the healthy
% individual, those in state $1$. All of the further dynamics are
% determined exogenously. However, $p$ does change between states, as it
% is the number of people in the final state divided by the total number
% of people. Thus we cannot simply take the right eigenvalues to compute
% the steady-state distribution of this matrix.

\section*{How many prevalence steady states are there?}

There always exists the trivial steady state, where $y_{ss} = 0$. If
there are no infected individuals, nobody can obtain the disease, and
individuals will copulate without fear of disease.

In steady state, individuals always make the same choices. $P^{*} =
P_{ss}$, $\alpha = \alpha_{ss}$ and $y_t = y_{t+1} = y_{ss}$.

Under these assumptions, the transition probability matrix is constant
over time in the steady state. We can apply the balance equations to
obtain the steady state distribution. This can be done by calculating
the left eigenvalues of the transition probability matrix, and taking
the eigenvector corresponding to $\lambda = 1$. Normalizing this eigenvector
gives us the stationary distribution. This stationary distribution is
given by

\begin{align*}
  \pi_1 &= \frac{r}{r+xr (1-P_{ss})^{\alpha_{ss}} + (1-P_{ss})^{\alpha_{ss}}}\\
  \pi_i &= \frac{r (1-P_{ss})^{\alpha_{ss}} }{r+xr (1-P_{ss})^{\alpha_{ss}}
            +(1-P_{ss})^{\alpha_{ss}} }\\
  \pi_{x+2} &= \frac{(1-P_{ss})^{\alpha_{ss}}}{r+xr (1-P_{ss})^{\alpha_{ss}}
            +(1-P_{ss})^{\alpha_{ss}} }
\end{align*}

Regardless of the value of $P_{ss}$, the probability of obtaining the
disease in steady state is, there is a single non-trivial steady
state. This Markov process is irreducible and therefore it makes sense
to compute the steady state in the way we have done.

There are always two steady states, the trivial steady state and the
disease-ridden steady state.

% For there to be a non-trivial steady state, the amount of people
% leaving a state needs to be equal to the amount of people entering a
% state. This is trivially true for all states except the first and the
% last. These conditions are known as the balance equations and they
% dictate that:

% % \begin{align*}
% %   ry_{ss} = 1-(1-y_{ss})^{\alpha_{ss}} + (1-r)y_{ss} \\
% %   r y_{ss} + (1-y_{ss})^{\alpha_{ss}} = 1 - (1-y_{ss})^{\alpha_{ss}}
%     %   \end{align*}
% \begin{align*}
%   r y_{ss} + (1-y_{ss})\left(1-P^{*}\right) &= (1-y_{ss})P^{*}\\
%   (1-y_{ss})\left(1-P^{*}\right) + (1-r)y_{ss} &= r y_{ss}
% \end{align*}

% These equations can be algebratized into:
% \begin{align*}
%   P^{*} &= \frac{1 - 2r y_{ss}}{1 - y_{ss}}\\
%   y_{ss} &= \frac{1}{3r -1}\\
% \end{align*}

% For this steady state to make sense, we need $3r+1 > 0$. 

% % This tells us the number of steady state solutions for $y_{ss}$. If $r
% % = \frac{1}{2}$, then the only solution is $y =
% % \frac{2}{3}$. Otherwise, however, there are two solutions for
% % $y_{ss}$.


% This tells us that there are $3$, or rarely, $2$ steady states to this system.

\section*{Can incentives explain why prevalence follows a cycle?}

The incentives in this problem are strange in the sense that once an
individual contracts the disease, he does not become contagious for
several periods. This means that the probability of being infected is
based in part by the probability of being infected in the past.

Note that changes in the probability of being infected disincentivise
the healthy individuals from fornicating, and incentivise the
contagious to procreate more. Those that do not have syphilis are at
more risk with a random fornication, and the contagious feel less
guilt, so they will procreate more. This acts to increase the
probability of being infected from a random encounter as a secondary
effect. Both of these combine to ensure that the costs of sexual
encounters increase for healthy individuals.

As the probability of being infected rose in the past, healthy
individuals changed their behavior to fornicate less. However, this
could not affect the amount of infected individuals for $x$
years. Imagine a shock occurs where there are suddenly $10$ percent
more infected people at a time $t$. Suddenly the probability of being
infected has risen, but for the next $x-1$ years, the people who were
infected previously are the only people who become contagious. The
growth of the system is unaffected until $x$ years later, where less
people become infected. This causes the chance of being infected to
decrease, but the behavior based on this decrease will not be shown in
the dynamics for $x$ years.

For the years until this behavioral change is made apparent, the
contagious rate will continue along its previous trajectory. 

This $x$ year delay distortion in the incentives will cause the
prevalence to cycle. Once the rate has decreased to a certain amount,
more people will copulate, and it will take $x$ years for the
prevalence to begin to increase. This will lead to another decrease
and so on.

The larger the value of $x$, the greater the cyclical behavior, as it
takes longer for changes in behavior to reflect back into the
incentives. Incentives always change $x$ years after the behavior
does. Therefore for there to be a cycle and any sense of delay, we
need $x$ to at least be $1$ or greater. If $x$ is zero, people are
able to react instantly, and the question is trivial.

\section*{How is your answer different if you recognize that people
  differ in terms of the number of partners that they have?}

Let us believe that individuals are heterogeneous across the benefit that
they receive from a fornication session. The individuals that receive
more benefit will choose to plunder more booty, and the individuals
that receive less benefit will choose to fornicate less. 

If people differ in the number of partners that they have, more
sexually active people are more likely to contract the disease. This
manifests itself as different choices of $\alpha$ for different
people. This means that the people more likely to contract the
disease, denoted $w$, are more likely to be the ones infected. As the
infection rate rises, these people are more disproportionately
infected. The remaining people are less likely to do deed already, and
seeing the higher infection rates, become even less likely to
fornicate.


This does not change the behavioral trends being cyclical. The effect
caused by heterogeneity is still delayed by $x$ time periods. The
expected number of healthy individuals that become diseased has
decreased, but any change in this number takes $x$ years to impact the
actual number of infected individuals. Regardless of whether or not
this change is effected by a change in the distribution of the healthy
individuals or the choices that the healthy individuals are
making. This simply makes the effects sharper within the cycles. As
the number of infected people rises, there is a larger decrease in the
expected quantity of hook-ups. This is caused by the behavioral
response just as in the homogeneous case, but also because the
remaining healthy people are less likely to hook up. This leads to a
larger decrease in sexual actions than under homogeneity. When
the infected number begins to drop, it will drop by more than if the
population was homogeneous, and likewise when it increases again.

There is no change in the qualitative behavior of the dynamic
system. Only the magnitude of deviations during each period have
changed, the behavior is still periodic with the same pattern of
increasing and decreasing. 



\section*{What would have been different about the prevalence series
  if prostitution had been legal?}

% We will treat prostitutes as incapable of screening their clients for
% diseases. In this framework, they can be viewed as individuals that
% have a very high number of partners. Let us model their behavior as
% unaffected by whether or not they have contracted the disease. They
% continue to be in business if they have contracted syphilis, and
% potential customers are unaware if they have syphilis, or the
% probability of prostitutes having syphilis. That is, their information
% has not changed. 

Let us believe that people are capable of screening prostitutes for
syphilis. This is the only real distinction that occurs when
legalizing prostitutes, as illegal prostitutes still exist
regardless. But now they have access to a legal system and rights to
ensure their ability to screen. This allows prostitutes to become a
substitute for hook-ups which presumably cost more money, but do not
run the risk of carrying the disease.

This means that when there are high amounts of syphilis, demand for
the prostitutes will be high, and when the disease is low, demand for
the prostitutes will be relatively lower.

Assuming that the benefit of fornication is equal between hookups and
prostitutes, the decision is made where the cost premium of a
prostitute is equal to the monetary cost of the disease times the
probability of being infected. Healthy individuals are then
indifferent between dating and using prostitutes, so there will be
some supply of prostitutes at the given price.

Individuals in general will face a lower risk of obtaining the
disease, as they are indifferent between a prostitute and a hook-up,
and there is some supply of prostitutes at the given price level. The
individuals that utilize the prostitutes will face no risk of the
disease where they once had a risk, so the overall level of the
disease will fall.

This will reduce the price of the prostitutes, decreasing the quantity
supplied. This will lead to less individuals using the prostitutes
than before, but still less diseases overall than when prostitution
was not an institution. We would still expect a cyclical behavior, but
the existence of prostitutes mutes its magnitude. There is an effect
of less prostitutes being available at low levels of the disease
causing the muting to be less than initially expected. However this
secondary effect cannot dominate the primary effect of there being
less disease-sharing hookups.

\section*{Does the emergence of AIDS help explain why syphilis prevalence dropped to new lows?}

Yes. The cost of a random hookup increases when there is another
disease in play. AIDS is also considered to have a much more negative
effect on the health of an individual than syphilis. This means that
individuals would be much more sensitive to hooking up with people, as
the cost of contracting AIDS was very high.

Each individual now faces some probability $\phi$ of obtaining AIDS from
a possible hookup. This cost is faced by all parties, as regardless of
having syphilis, one can still contract AIDS. As there is no cure for
AIDS, the dynamics do not allow for people to become healthy again,
and the present cost of obtaining that disease is much higher.  Denote
this cost by $\Theta$

\begin{align*}
  \alpha &= \argmax_{\alpha} \quad u(\alpha) - \delta\alpha - \beta( 1 - (1-p)^{\alpha}) - \Theta ( 1 - (1-\phi)^{\alpha})\\
  a_c &= \argmax_{a_c} \quad u( a_c) - \delta a_c - \alpha ( 1 - p)\gamma - \Theta ( 1 - (1-\phi)^{a_c})\\
  a_n &= \argmax_{a_n} \quad u( a_n) - \delta a_n  - \Theta ( 1 - (1-\phi)^{a_n})
\end{align*}

We can see that the cost for all groups has increased, so there will
be less quantity of hook ups provided by all three groups. We cannot
justify that the probability of obtaining the disease from a single
hook-up has increased. Since the utility function is concave,
non-contagious infected individuals decrease their fornication the
most. It may be that healthy individuals reduce their hook ups less
than contagious individuals, but either way the probability of
obtaining the disease is indeterminate.

Though it is likely that the decrease in the amount of copulation by
the healthy individuals would offset this lowered probability, it
cannot be shown to occur in all cases. Consider very concave utility
functions, high costs of syphilis, AIDS, and guilt. This would mean
that the non-contagious individuals reduce their fornication by a
large amount, and the other two groups reduce it by little. Now the
probability of obtaining the disease from a random hook-up is much
higher. This would lead to a higher amount of syphilis, despite every
group desiring less hook ups. 

We do know that the amount of copulation has decreased for a
certainty. However, because every group responds to this added cost of
possibly obtaining AIDS in a different way, it can be that healthy
individuals are more at risk of obtaining the disease. The existence
of non-contagious infected individuals allowed for healthy individuals
to be shielded from the contagious individuals, but since that group
responds most strongly to the presence of AIDS, healthy individuals
can be at more risk.

However, if all groups respond near the same proportion, we would
expect the presence of AIDS to reduce the prevalence of syphilis, as
all groups change their amount of copulation, and the probability of a
healthy individual obtaining the disease would not change by much. In
this case, syphilis cases reduce, and we get the behavior noted in the
time series data give.


\end{document}
