\documentclass[12pt]{paper}

\usepackage[T1]{fontenc}

\usepackage[margin=1in]{geometry}
\usepackage{amsmath}
\usepackage{bm}
\usepackage{amsthm}
\usepackage{mathtools}
\usepackage{amsfonts}
\usepackage{bbm}
\usepackage{graphicx}
\usepackage{minted}

\usepackage{longtable}
\usepackage{wrapfig}
\usepackage{rotating}
\usepackage[normalem]{ulem}
\usepackage{amsmath}
\usepackage{textcomp}
\usepackage{amssymb}
\usepackage{capt-of}
%\usepackage{hyperref}
\usepackage{tikz}
\usepackage{bm}
\usepackage{minted}

\usepackage{amsmath}
\usepackage{bm}
\usepackage{amsthm}
\usepackage{mathtools}
\usepackage{amsfonts}
\usepackage{bbm}
\usepackage{graphicx}

\usepackage{fontspec}
\setmonofont{Ubuntu Mono}


\DeclareMathOperator{\diam}{diam}
\newcommand{\Mod}[1]{\ \mathrm{mod}\ #1}
%\DeclareMathOperator{\mod}{mod}
\DeclareMathOperator{\interior}{int}
\DeclareMathOperator{\close}{cl}

\newcommand{\met}[1]{d \left ( #1 \right )}
\newcommand{\brak}[1]{ \left [ #1 \right ] }
\newcommand{\cbrak}[1]{ \left \{ #1 \right \}}
\renewcommand{\vec}[1]{ \bm{ #1 }}
\newcommand{\abs}[1]{\left \lvert #1 \right \rvert}
\newcommand{\seq}[1]{{\left \{ #1 \right \}}}
\newcommand{\conj}[1]{ \overline{ #1 } }
%\newcommand{\close}[1]{ \bar{ #1 } }
\newcommand{\set}[1]{\left \{ #1 \right \}}
\newcommand{\Lim}{\lim\limits}
\newcommand{\compose}{\circ}
\newcommand{\inv}[1]{{#1}^{-1}}
\newcommand{\compl}[1]{{#1}^{c}}



\newcommand{\setR}{ \mathbb{R} }
\newcommand{\setQ}{ \mathbb{Q} }
\newcommand{\setZ}{ \mathbb{Z} }
\newcommand{\setN}{ \mathbb{N} }

\newcommand{\plim}{ \overset{p}{\to} }
\newcommand{\mean}[2][N]{ \overline{ #2 }_{#1}}
\newcommand{\exV}[1]{\mathbb{E} \left [ #1 \right ]}
\newcommand{\Vari}[1]{\mathbb{V} \left ( #1 \right )}

\newcommand{\est}[2][n]{ \widehat{ #2 }_{#1}}
\newcommand{\altest}[2][n]{ \tilde{ #2 }_{#1}}

\newcommand{\indicate}[1]{ \mathbbm{1}_{\{#1\}}}
\newcommand{\convDist}{ \overset{d}{\to}}
\newcommand{\unif}{\emph{U}}
\newcommand{\normal}{\mathcal{N}}
\newcommand{\eye}{\mathbbm{I}}

\newcommand{\bigO}{\mathcal{O}}
\newcommand{\Lagrange}{\mathcal{L}}

\newcommand{\deriv}[2]{\frac{ \partial #1}{ \partial #2}}

\DeclarePairedDelimiter{\ceil}{\lceil}{\rceil}
\DeclarePairedDelimiter{\floor}{\lfloor}{\rfloor}
\DeclarePairedDelimiter{\norm}{\lVert}{\rVert}

\title{Optimization Conscious Econometrics pset 4}
\author{Timothy Schwieg}
\begin{document}

\maketitle

\section{3.18}

Consider the Simplex method applied to a standard form problem and
assume that the rows of the matrix $A$ are linearly independent. For
each of the statements that follow, give either a proof or
counterexample.


\begin{enumerate}
\item An iteration of the simplex method may move the feasible
  solution by a positive distance while leaving the cost unchanged.

  \vspace{.15in}
  
  In the second step of the simplex method, the reduced costs of
  moving in every direction is computed. For the cost to be unchanged,
  the reduced cost in the direction chosen must be zero. However the
  algorithm terminates if they are all non-negative, and otherwise
  chooses a strictly negative element of the reduced cost. This means
  that it will never choose a negative reduced cost.

  Note that if it does not move a positive distance (i.e. it is
  degenerate), this will leave the cost unchanged.

\item A variable that has just left the basis cannot reenter in the
  very next iteration.

  \vspace{.15in}
For a variable to re-enter the basis, it would require that its
reduced cost be negative. However, when a variable leaves the basis,
that is because we traveled in a direction that it was the first to
hit $0$. Thus we increased the objective by decreasing the value of
$x_{\ell}$. It cannot be that increasing the value of $x_{\ell}$ from that
position somehow decreases the value of the objective function. We
must move to a different index for this to be possible.  That is, it
can possibly be up for re-entry on any iteration except the one where
it was immediately removed. 

  
  
\item A variable that has just entered the basis cannot leave in the
  very next iteration.

  \vspace{.15in}

  This is false. Consider a simplex determined by a right triangle at
  the origin. If the reduced cost is negative in any movement from the
  origin, it is possible that it moves from the origin to the
  right-most point, then to the top-most point. In this, the $x$
  dimension enters the basis, then immediately leaves in favor of $y$.
  
  
\end{enumerate}

\section{4.4}

Let $A$ be a symmetric square matrix. Consider the linear programming
problem:

\begin{align*}
  \min \quad & c'x\\
  \text{s.t. } & Ax \geq c\\
  & x \geq 0
\end{align*}

Prove that if $x^{*}$ satisfies $Ax^{*} = c$ and $x^{*} \geq 0$ then
$x^{*}$ is an optimal solution.

\vspace{.3in}

The objective function evaluated at $x^{*}$ is:
$x^{*\prime} A' x^{*}$.  Assume that this is not optimal for the linear
program. Let $y$ be optimal for the problem. Since $y \neq x^{*}$ then
$Ay \geq c$ with the inequality being strict in at least one
component. This can be written as $Ay = c + \epsilon$ where
$\epsilon \geq 0$, and it is strict in at least one component.

From the definition of $y$ being a minimizer and $y \neq x^{*}$ and using
$c = Ax^{*}:$ and the symmetry of $A$:

\begin{align*}
  c' x^{*} &> c' y\\
  x^{*\prime} A' x^{*} &> x^{*\prime} A' y\\
  x^{*\prime} A x^{*} &> x^{*\prime} A y\\
  x^{*\prime} c &>  x^{*\prime} ( c + \epsilon)\\
  0 &> x^{*\prime} \epsilon
\end{align*}

However we know that $x^{*} \geq 0$ and $\epsilon \geq 0$, so it must be the case
that $x^{*\prime}\epsilon \geq 0$. This is a contradiction, and therefore we find
that $y = x^{*}$. Thus $x^{*}$ must be optimal.

\section{4.26}

Show that exactly one of the following holds:

\begin{enumerate}
\item There exists an $x \neq 0$ such that $Ax = 0, x \geq 0$
\item There exists some $p$ such that $p'A > 0$
\end{enumerate}

\vspace{.3in}

Examine condition (1). For there to be a vector $x \neq 0$ such that $Ax
=0$, then the null space of $A$ must have dimension greater than
zero. This means that some column can be written as a linear
combination of the previous columns, but even stronger than this, we
require that the weights used in the linear combination to be
non-negative.

Let $a_1,...,a_m$ denote the columns of $A$, then this means that $c_1
a_1 + ... + c_m a_m = 0$ and $c \geq 0$. This implies that some column
$j$ satisfies $a_j = - \sum_{i \neq j} c_i' a_i$. where $c_i' \geq 0$.

Assume that $A$ satisfies condition (1) and suppose that it satisfies
condition (2). Then $p'a_i > 0 \quad i = 1,...,N$ However, we know that
$p'a_j = - \sum_{i \neq j}c_i' p'a_i$. Since $c_i \geq 0$ and $p'a_i > 0$, this
term must be negative, which contradicts $p'a_i > 0$. Thus we cannot
have both condition $1$ and condition $2$ holding.

Consider the set $S = \{ Ax, x \geq 0 \}$. Assume that $0 \notin S$.
$S$ is closed, nonempty and convex. By the separating hyper-plane
theorem, there exists a vector $p$ such that $p'0 < p'Ax$ where
$x \geq 0$. Applying this multiple times, taking $x$ to be each of the
standard ordered basis, we arrive at $p'0 = 0 < p'A$. Therefore if
condition (1) is not met, it must be the case that condition (2) is
met.

This implies that we cannot have neither conditions holding, and since
we proved above that we cannot have both conditions, the logical
conclusion is that only one of the conditions may hold.


\section{ Part 2}

\subsection{1}

The code for the revised simplex method is below:

\begin{minted}[frame=lines,fontsize=\scriptsize,xleftmargin=\parindent,linenos,mathescape,breaklines=true,stripnl=true,firstnumber=last]{julia}  
function RevissedSimplexIteration( BInv::Matrix{Float64}, x::Vector{Float64},
                           A::Matrix{Float64}, c::Vector{Float64},
                           b::Vector{Float64}, k::Vector{Int64},
                           M::Int64, N::Int64)


    #First let us compute some costs, we stop computing costs as soon
    #as we have a negative cost
    j = -1
    #Numerical Precision problems when working with the inverse
    for i in 1:M
        if( c[i] - dot( c[k], BInv*A[:,i]) < -1e-8)
            j = i
            break
        end
    end
   
    #Check if we are at an optimal solution
    if( j == -1 )
        return 1
    end
    
    u = BInv*A[:,j]
    #Check if the problem is unbounded below
    if( sum(u[i] > 0 for i in 1:M ) == 0)
        return -1
    end
    

    #This is an implementation of Bland's Rule
    min = 1.0e10
    ℓ = -1
    for i in 1:M
        if( u[i] > 0.0 )
            thetaTemp = (x[k[i]] / u[i])
            #The strict inequality means that the first i wins the
            #tie.
            if( thetaTemp < min )
                min = thetaTemp
                ℓ = i
            end
        end
    end
    #Now the basis has k instead of ℓ
    k[ℓ] = j

    #Are elementary matrix operations faster than this?
    for i in 1:M
        if i != ℓ
            @inbounds BInv[i,:] -= (u[i] / u[ℓ])*BInv[ℓ,:]
        end
    end
    val = u[ℓ]
    for j in 1:M
        @inbounds BInv[ℓ,j] =  BInv[ℓ,j] / u[ℓ]
    end
    
    x .= 0
    x[k] = BInv*b
    
    #Continue iterating:
    return 0
end

\end{minted}


For a simulation, since generating random matrices $A$, $c,b$ is not
very feasible, as there is no coded version of phase-II which obtains
an initial basic feasible solution. Without a basic feasible solution,
we wish to simulate problems in which there is a known basic feasible
solution always, hopefully starting at zero.

To this end, I simulate the problem using different cases of quantile
regression. In particular, I use $\tau = .5$ so that the Least-Absolute
Deviations estimator can be used, as it is easier to simulate with
thoughts towards a more robust calculation of the mean effect.

The true coefficients of the model are generated randomly, coming from
the Uniform (-10,10), where there is no constant. I used five
coefficients, and simulated data for different amounts of
data. Run-times were averaged over many iterations in an attempt to
capture run-time speed compared to speed costs related to allocation
of memory that become relevant for even medium sized data sets.

Memory is important because there are $M$ rows in the $A$ matrix, and
then there are  $2M + 2p$ columns. This means that the storage
requirements are $\bigO( M^2 )$ and allocation becomes quite costly
regardless of the speed of the algorithm. This could be ameliorated
through better coding, but did not seem relevant for this
assignment. The Table below summarizes the run times:

\vspace{.1in}

\begin{centering}
\begin{tabular}{|l|l|l|}\hline
  M & Simplex Run Time & Revised Run Time\\\hline
250 &
0.300426 &
0.750987667\\
500 &
5.438479 &
10.307776333
\\
750 &
25.473887 &
39.933402333
\\
1000 &
21.178596667 &
28.797609333
\\
1250 &
75.5935325 &
102.380060\\\hline
\end{tabular}  
\end{centering}

\vspace{.1in}

It is important to note that the difficulty to converge of the
algorithm is driven more by the shape, which was dictated by the
randomly simulated data, both algorithms were run against the same
data, so while both struggled with the draw at $M = 750$, the relative
performance differences between the two can be looked at to draw
conclusions.


One would expect the effects to be increasing in the model complexity,
as the matrix $B$ becomes more and more costly to invert, a
$\bigO(N^3)$ operation. But this is not the case. For relative small
values of $M$, the revised algorithm has a larger return, netting half
the run time of the regular inversion technique. As the problem
scales, this effect is still prominent, but begins to recede to only
being three quarters of the run time of the regular algorithm.

Some of these differences have to be credited to the inversion
techniques used in Julia, which exploit many of the structures of the
matrix quite well, while my column operations are clumsy column-wise
operations at best. It may be that generating elementary matrices and
multiplying could exploit this structure better, but I was unable to
find prefabricated elementary matrix code to test this hypothesis. It
is clear that in my implementation, while there are performance gains
from using the revised simplex method, there does not appear to be a
reduction in the order of the complexity of the problem from
maintaining the inverse $B$ matrix via row operations. 



\subsection{2}



Show equivalence of the two dual formulations of the quantile
regression problem.

\begin{align*}
  \max_d \quad &Y'd\\
  \text{ subject  to: } & X'd = 0\\
  & (\tau-1)1_n \leq d \leq \tau 1_n
\end{align*}

and

\begin{align*}
  \max_a \quad Y'a\\
  \text{ subject to: } & X'\alpha = (1-\tau)X'1_n\\
  & \alpha \in [0,1]^n
\end{align*}

\vspace{.3in}

The second problem may be rewritten as:


\begin{align*}
  \max_a \quad Y'a\\
  \text{ subject to: } & X'(\alpha - (1-\tau)1_n) = 0\\
  & \alpha \in [0,1]^n
\end{align*}

Allowing $d = \alpha - (1-\tau)1_n$ we see that the constraint that $\alpha \in
[0,1]^n$ is the same as $d \geq -(1-\tau)1_n$ and $d \leq 1 - (1-\tau)1_n$ This
can be summarized as:

\begin{align*}
  (\tau-1)1_n \leq d \leq \tau 1_n
\end{align*}

Therefore the optimization question becomes:

\begin{align*}
  \max_d \quad Y'd\\
  \text{ subject to: } & X'd = 0\\
  & (\tau-1)1_n \leq d \leq \tau 1_n
\end{align*}

This is exactly the initial formulation, so the problems must be
equivalent.

\subsection{3}

A box constrained problem can always be rewritten in a dimension of
$2n$ where there is both an $x$ and a slack variable $s$ such that
$x+s =1$ and $x,s \geq 0$. By nature of this constraint, and the fact
that we are iterating across basic feasible solutions, either $x_i$ or
$s_i$ will be in the basis, with the possibility of both.

In problem (2) there are $n+m$ equality constraints, and $2n$
parameters, as we need a slack for each parameter. There will then be
$n-m$ parameters in the active basis at any time.

Let us begin with some basic feasible solution, and a working version
of $\inv{B}$. Compute $\bar{c}_j = c_j - c_B' \inv{B} A_j$. If
$\bar{c}_j \geq 0$ then we have an optimal solution, if not, choose some
$j$ such that $\bar{c}_j < 0$

Now, we wish to have $A_j$ enter the basis. Since it impossible for
there to be neither $x_i$ or $s_i$ in the basis, the only elements
that need to be considered for removal from the basis are the terms
with $x_i,s_i$ both active, or the $j + n \Mod 2n$ term.

To this end we compute $u = \inv{B}A_j$ as before. However in the
computation of $\theta$, we need only consider the elements above.

\begin{equation*}
  \theta^{*} = \min_{i, u_i > 0 \cap \{ i, i+n \Mod 2n \in B\: \cup\: i = n+j \Mod 2n\}} \frac{x_{B(i)}}{u_i}
\end{equation*}

We then replace the basis as before, letting $\ell$ fulfill the above
minimization problem, and computing the values of $\inv{B}$ using the
same elementary row operations.

In psuedo-code form, this algorithm means following the instructions:

Assuming that we begin with some basic feasible solution.
\begin{enumerate}
\item Compute $\bar{c}_j$ as above, if it is all non-negative, then
  the solution is optimal.
\item Choose some $j$ such that $\bar{c}_j < 0$.
\item Compute $u - \inv{B} A_j$
\item For each element of $u$:
  \begin{enumerate}
  \item If $u_i \leq 0$, skip this element.
  \item If $i + n \Mod 2n$ is not in the basis $B$, and $i \neq n+j \Mod
    2n$, skip this element.
    
  \item Find the argmin of $\frac{x_{B(i)}}{u_i}$, call this index
    $\ell$.
  \end{enumerate}
  
\item Exchange $\ell$ for $j$ in the basis, updating the inverse matrix
  using the appropriate elementary matrix operations.
\item Compute $x_B = \inv{B} b$, and set $x_{-B} = 0$.
\end{enumerate}

\subsection{4}

The code for the Barrodale and Roberts algorithm is given below.

\begin{minted}[frame=lines,fontsize=\scriptsize,xleftmargin=\parindent,linenos,mathescape,breaklines=true,stripnl=true,firstnumber=last]{julia}
function Barrodale( X::Matrix{Float64}, x::Vector{Float64},
                    k::Vector{Int64}, b::Vector{Float64},
                    c::Vector{Float64}, BInv::Matrix{Float64},
                    p::Int64, M::Int64, N::Int64)

    cBar = Vector{Float64}(undef,2*p)

    ℓ = 0
    u = Vector{Float64}(undef,0)

    for i in 1:p
        j = -1
        min = 1e10
        #Which β should enter the distribution?
        for z in 1:p
            cBar[z] = c[z] - dot( c[k], BInv*X[:,z])
            if cBar[z] < min
                j = z
                min = cBar[z]
            end
        end
        for z in (p+1):2*p
            cBar[z] = c[z] - dot( c[k], -BInv*X[:,z-p])
            if cBar[z] < min
                j = z
                min = cBar[z]
            end
        end
        #j is now the smallest element of cBar

        if j <= p 
            ℓ,u = ChangeSignPivots( c, BInv, X[:,j], x, k, b, p, M, j)
        else
            ℓ,u = ChangeSignPivots( c, BInv, -X[:,j-p], x, k, b, p, M, j)
        end

        # Now we do a normal pivot bringing in β[j]
        k[ℓ] = j

        #Are elementary matrix operations faster than this?
        for z in 1:M
            if( z == ℓ)
                continue
            end
            BInv[z,:] -= (u[z] / u[ℓ])*BInv[ℓ,:]
        end
        BInv[ℓ,:] ./=  u[ℓ]

        #Compute the new x value
        x .= 0.0
        x[k] = BInv*b
    end
    #Phase 1 complete.
    println( x[1:2*p])

    cBar = Vector{Float64}(undef,2*M)
    min = 1e10
    j = -1
    for i in 1:M#(2*p+1):(2*p+M)
        cBar[i] = c[i+2*p] - dot( c[k], BInv[:,i])
        if cBar[i] < min
            j = i+2*p
            min = cBar[i]
        end
    end
    for i in 1:M#Note that the second half of the residuals uses -I 
        cBar[M+i] = c[M+i+2*p] - dot( c[k], -BInv[:,i])
        if cBar[i] < min
            j = i+2*p+M
            min = cBar[M+i]
        end
    end

    # We stop once all reduced costs are positive.
    while( min < 0.0)
        #Since we know we are using A[:,j] where j is a standard
        #ordered basis, we just need to make sure we get the element
        #correct. Lots of silly modular shit to do that
        sob = zeros(M)
        sob[((j-2*p-1)%M)+1] = 1.0-2.0(j-2*p > 2*p+M)
        ℓ,u = ChangeSignPivots( c, BInv, sob, x, k, b, p, M, j)
        # Now we do a normal pivot bringing in β[j]
        k[ℓ] = j

        #Are elementary matrix operations faster than this?
        for z in 1:M
            if( z == ℓ)
                continue
            end
            BInv[z,:] -= (u[z] / u[ℓ])*BInv[ℓ,:]
        end
        BInv[ℓ,:] ./=  u[ℓ]

        #Recalculate all of the reduced costs for u,v
        min = 1e10
        j = -1
        for i in 1:M#(2*p+1):(2*p+M)
            cBar[i] = c[i+2*p] - dot( c[k], BInv[:,i])
            if cBar[i] < min
                j = i+2*p
                min = cBar[i]
            end
        end
        for i in 1:M#(2*p+1):(2*p+M)
            cBar[M+i] = c[M+i+2*p] - dot( c[k], -BInv[:,i])
            if cBar[i] < min
                j = i+2*p+M
                min = cBar[M+i]
            end
        end
    end

    #Compute the new x value
    x .= 0.0
    x[k] = BInv*b
    return x
end

function ChangeSignPivots( c::Vector{Float64}, BInv::Matrix{Float64},
                           Aj::Vector{Float64}, x::Vector{Float64},
                           k::Vector{Int64}, b::Vector{Float64},
                           p::Int64, M::Int64, j::Int64)

    exiting = 0
    ℓ = -1

    u = Vector{Float64}
   
    while( c[j] - dot( c[k], BInv*Aj) < 1e-8)
        u = BInv*Aj
        min = 1.0e10
        ℓ = -1
        for z in 1:M
            if( u[z] > 0.0 )
                thetaTemp = (x[k[z]] / u[z])
                #The strict inequality means that the first i wins the
                #tie.
                if( thetaTemp < min )
                    min = thetaTemp
                    ℓ = z
                end
            end
        end

        if ℓ == -1
            println("ℓ issue")
            return -1, u
        end

        #Now we know we want k[ℓ] to exit, we need to find its
        #negative (positve) counter-part
        exiting = k[ℓ]
        # If we need to do a beta change of sign
        if exiting <= 2*p
            # Note that Julia is 1-indexed so for modular
            # arthimetic we need to minus one, mod, then add one
            entering = (exiting + p-1) % (2*p)+1
        else
            entering = ((exiting-2*p)+M-1)%(2*M) + 2*p+1
        end
     
        k[ℓ] = entering
        BInv[ℓ,:] *= -1
        x .= 0.0
        x[k] = BInv*b
    end
    #If we had a negative Reduced cost before, we should undo the
    #last change, and do a regular pivot. These pivots are cheap,
    #so doing 2 more pivots than we need isn't a big deal
    k[ℓ] = exiting
    BInv[ℓ,:] *= -1

    return  ℓ,u
end
\end{minted}


In order to test the performance of my simulation against the Gurobi
implementation, I use the same simulation specifics as in question
1. This allows for an easy comparison between my naive-simplex
approach, the Barrodale-Roberts coded algorithm, and the Gurobi ``big
boy'' software.

In order to access Gurobi, I use Julia for Mathematical Programming,
which functions as a solver interface much like AMPL. The code used to
answer the problem is given below:

\begin{minted}[frame=lines,fontsize=\scriptsize,xleftmargin=\parindent,linenos,mathescape,breaklines=true,stripnl=true,firstnumber=last]{julia}
function GurobiQuantReg( X::Matrix{Float64},
                         Y::Vector{Float64},
                         τ::Float64,
                         p::Int64, M::Int64, N::Int64)

    
    m = Model(solver = GurobiSolver())
    @variable( m, β⁺[1:p] >= 0)
    @variable( m, β⁻[1:p] >= 0)
    @variable( m, u[1:M] >= 0)
    @variable( m, v[1:M] >= 0)
    @constraint( m, fit[i=1:M],
                 sum( X[i,j]*β⁺[j] for j in 1:p )
                 - sum( X[i,j]*β⁻[j] for j in 1:p)
                 + u[i] - v[i] == Y[i] )
    @objective( m, Min, τ*sum( u[i] for i in 1:M )
                + (1-τ)*sum(v[i] for i in 1:M) )
    status = solve(m)
    println(getvalue(β⁺))
    println(getvalue(β⁻))
    return [getvalue(β⁺), getvalue(β⁻), getvalue(u), getvalue(v)]
end
\end{minted}

The results of the horse-race are summarized below.
The professional solver is orders of magnitude faster than even the
Barrodale-Roberts algorithm. It is worth noting that the results from
Gurobi are computed from using a seperate machine which was
considerably slower than the others, so this table understates the
advantages of using Gurobi.

\begin{centering}
\begin{tabular}{|l|l|l|l|l|}
  \hline
  M & Revised Simplex & Regular Simplex & Barrodale Roberts & Gurobi\\\hline
250 &
0.300426 &
0.750987667 &
0.292047333 &
0.034217
\\
500 &
5.438479 &
10.307776333 &
0.473022 &
0.065429
\\
750 &
25.473887 &
39.933402333 &
2.173836 &
0.078337
\\
1000 &
21.178596667 &
28.797609333 &
2.868348 &
0.156235
\\
1250 &
75.5935325 &
102.380060 &
19.424499667 &
0.170933
\\\hline
\end{tabular}
  
\end{centering}
The same mechanics of using the same random seed and data generation
are held constant across the four methods. Much of the cost of these
methods can be attributed to the construction of the A matrix. The
Barrodale-Roberts and Gurobi methods do not require that this be
constructed and stored in memory, so there is a large gain in terms of
allocation and garbage collection. However it is clear that there is
more to this than simply that, as we see enormous differences between
the methods.

For the Gurobi solver, the number of iterations was always 1-5
iterations more than the number of data points. It is difficult to
contrast the Barrodale-Roberts algorithm to this, as it made many
inefficient but cheap iterations in the form of change of sign pivots,
so counting iterations may not be a clear way of measuring its
effectiveness. However as the computation time suggests, it is still
significantly slower than the professional solver that is able to
further exploit the structure of the model. It is clear that there are
enormous gains from using commercial software, though the
Barrodale-Roberts algorithm performs extremely well when compared
against a traditional simplex algorithm, it cannot attempt to compete
with serious proprietary software maintained by professionals. 


  


\end{document}
