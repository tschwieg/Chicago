\documentclass[12pt]{paper}

\begin{document}
\section{2}

Since the $\epsilon_{ti}$ is private information of the firm, but the
parameters are common knowledge, the decision of firm 1 depends on the
unknown parameter of firm 2, and vice versa.

Firm one will enter when
\begin{equation*}
  \alpha x_t + \epsilon_{1t} - \delta y_{2t} \geq 0
\end{equation*}

In this case, it makes profits equal to that quantity, if it chooses
not to enter, it makes profits of $0$.

\begin{align*}
  \exV{\Pi} &= y_{1t} \brak{\alpha x_t + \epsilon_{1t} - \delta \exV{y_{2t}}}\\
  &= y_{1t} \brak{\alpha x_t + \epsilon_{1t} - \delta \Pr( y_{2t} = 1)}
\end{align*}



Firm 1 wishes to maximize its expected profits, so it will not choose
to enter if it expects negative profits, and will enter if the profits
are positive. This indicates that its strategy is just to enter when
$\exV{\Pi} \geq 0$. Since there is no difference in the parameters between
firms, except for $\epsilon$, this is true for firm 2 as well.
\begin{equation*}
  y_{it} = \indicate{ \alpha x_t + \epsilon_{it} - \delta \Pr( y_{-it} = 1) \geq 0}
\end{equation*}

Taking expectations we get
\begin{align*}
  \exV{y_{it}} &= \Pr( \epsilon_{it} \geq \delta \Pr( y_{-it} = 1) - \alpha x_t )\\
  \Pr( y_{it} = 1) &= 1 - F_{\epsilon}( \delta \Pr( y_{-it} = 1) - \alpha x_t)
\end{align*}

Applying the Distributional assumption that the $\epsilon$ are distributed
Type-1 extreme value, we get that:

\begin{align*}
  \Pr( y_{1t} = 1) &= \frac{\exp(\delta \Pr( y_{2t} = 1) - \alpha x_t)}{1 +
                     \exp(\Pr(y_{2t} = 1) - \alpha x_t)}\\
  \Pr( y_{2t} = 1) &= \frac{\exp(\delta \Pr( y_{1t} = 1) - \alpha x_t)}{1 +
                     \exp(\Pr(y_{1t} = 1) - \alpha x_t)}
\end{align*}

The solution to these two equations is the Bayes-Nash equilibrium
probabilities that each firm enters the market. Firm i just believes
that firm j enters the market with these probabilities, and then
maximizes his profit under that assumption, while firm j does the
same. This will constitute a Bayes-Nash equilibrium.

As a result of this, from firm j's point of view, firm i is simply
flipping a coin with probability given by the solution to these
equations. When that coin lands on heads, he enters the market, and
when it is tails he chooses not to enter the market. 


\section{3}

From the Econometricians' point of view, the only randomness in the
choice of firm $i$ is his own draw in $\epsilon_{it}$ as he has already taken
an expectation of the draw of firm $j$ and the assumption of
Bayes-Nash equilibrium constrains these beliefs to satisfying the
above system. 


\end{document}
