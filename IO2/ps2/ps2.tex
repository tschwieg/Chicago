\documentclass[12pt]{paper}

\usepackage{Schwieg}
\usepackage[margin=1in]{geometry}

\title{Advanced Industrial Organization 2 Pset 2}
\author{Timothy Schwieg \and Paulo Henrique Ramos}

\begin{document}
\maketitle

\section{Berry, Levinsohn and Pakes (1995)}

\subsection{1}



The firms observe the characteristics of their products, as well as
the characteristics of the products of all of their competitors. The
prices in the market are observed by all. Only the $\epsilon_{ijt}$ remains
unobserved by the firms. 

A consumers utility is given by:
\begin{equation*}
  u_{ijt} = \alpha_i p_{jt} + x_{jt} \beta_i + \xi_{jt} + \epsilon_{ijt}
\end{equation*}

Let $\epsilon_{ijt} \sim T1EV$. Note that this distribution is in the
location-scale family, so we may add deterministic components to it
and only affect the mean.

Note that the maximum of a set of $T1EV ( \alpha_j)$ random variables is
distributed $T1EV\left[ \log \sum_{i=1}^n \exp (\alpha_i) \right]$.

\begin{align*}
  \Pr( i \rightarrow j) &= \Pr( u_{ijt} > u_{ikt} \quad \forall k \neq j)\\
  &= \Pr( \alpha_i p_{jt} + x_{jt} \beta_i + \xi_{jt} + \epsilon_{ijt} > \alpha_i p_{kt} + x_{kt}
    \beta_i + \xi_{kt} + \epsilon_{ikt} \quad \forall k \neq j )\\
  &= \Pr( \alpha_i p_{jt} + x_{jt} \beta_i + \xi_{jt} + \epsilon_{ijt} > \max_{k \neq j} \{ \alpha_i p_{kt} + x_{kt}
    \beta_i + \xi_{kt} + \epsilon_{ikt}\} )\\
\end{align*}

From the above note, we know that

\begin{equation*}
  \max_{k \neq j} \{ \alpha_i p_{kt} + x_{kt} \beta_i + \xi_{kt} + \epsilon_{ikt}\} \sim T1EV(
  \log \sum_{k \neq j} \alpha_i p_{kt} + x_{kt} \beta_i + \xi_{kt}) 
\end{equation*}

For notational convenience, call this object $u_{i,-j,t}$. We may also
note that
\begin{equation*}
  \alpha_i p_{jt} + x_{jt} \beta_i + \xi_{jt} + \epsilon_{ijt} \sim T1EV( \alpha_i p_{jt} + x_{jt} \beta_i + \xi_{jt} )
\end{equation*}

Lastly, the difference between two independent $T1EV$ distributions is
distributed logistically. Since $\epsilon_{ijt}$ are all independent, and our
transformations are simply adding to the location of each of the
distributions, they will be independent as well.

\begin{align*}
  \Pr( \alpha_i p_{jt} + x_{jt} \beta_i + \xi_{jt} + \epsilon_{ijt} > u_{i,-j,t} )
  &= \Pr( u_{i,-j,t} -\alpha_i p_{jt} +x_{jt}\beta_i +\xi_{jt} +\epsilon_{ijt} \leq 0)\\
  &= \frac{\exp(\alpha_i p_{jt} + x_{jt} \beta_i + \xi_{jt})}{\sum \exp(\alpha_i p_{kt} +
    x_{kt}\beta_i + \xi_{kt})}
\end{align*}

This is the probability that individual $i$ elects to purchase good $j$
when the prices are the vector $p$ and the characteristics are $x$ and
$\xi$.

Assume that there is a unit mass of consumers, whose preferences are
such that
$(\alpha_i, \beta_i')' \sim \normal( \theta_1, diag(\theta_2^2))$. The demand for good
$j$ in market $t$ is the expected number of consumers who choose good
$j$. This can be computed as the integral over the distribution of
parameters of the probability that consumer $i$ chooses good $j$.

\begin{equation*}
  D_{jt}(p,x,\xi,\theta) = \int_{\setR^4} \frac{\exp(\alpha p_{jt} + x_{jt} \beta +
    \xi_{jt})}{\sum \exp(\alpha p_{kt} + x_{kt}\beta + \xi_{kt})} dF(\alpha,\beta)
\end{equation*}

\subsection{2}

Ain't nobody got time for dis

\subsection{Estimation of the BLP}

Rather than Monte-Carlo Integration, which has relatively poor
convergence rates, we shall employ Gauss-Hermite quadrature. As the
dimension of the integral is only four, this is a reasonable exercise,
though for larger dimensions it would be better to choose Monte-Carlo
Integration.

By this choice of quadrature, there is no error introduced by the
random sampling from the integral. The Calculations are as follows:

It is known that
\begin{equation*}
  \Pr( i \rightarrow j) = \frac{\exp( \alpha_i p_j + x_j' \beta_i + \xi_j)}{\sum_{k \in
      \mathcal{F}_t} \exp( \alpha_i p_k + x_k' \beta_i + \xi_k)}
\end{equation*}

The distribution of $\alpha,\beta$ is given by:

\begin{equation*}
  (\alpha,\beta) \sim \normal( \theta_1, diag(\theta_2)^2)
\end{equation*}
Where $\theta_1,\theta_2$ are vectors of the expected value, and standard deviations
accordingly. Since we see that the covariance between each of these is
zero, we know that these are independent random variables.

We do know that since the normal distribution is in the location-scale
family, we may rewrite the definition of $(\alpha,\beta)$ as:

\begin{equation*}
  (\alpha,\beta) = \theta_1 + diag(\theta_2) Z_p
\end{equation*}

Where $Z_p$ is a p-dimensional vector of independent standard normal
random variables. This fact will be useful for the numerical
integration. Taking the expectation of the original expression leads
to:

\begin{align*}
  \exV{\Pr( i \rightarrow j)} &= \int\frac{\exp( \alpha p_j + x_j' \beta + \xi_j)}{\sum_{k \in
                      \mathcal{F}_t} \exp( \alpha p_k + x_k' \beta + \xi_k)} dF(\alpha,\beta)\\
                    &=  \int \int \int \int\frac{\exp( \alpha p_j + x_j' \beta + \xi_j)}{\sum_{k \in
                      \mathcal{F}_t} \exp( \alpha p_k + x_k' \beta + \xi_k)} dF_{\alpha}
                      dF_{\beta_1} dF_{\beta_2} dF_{\beta_3}
\end{align*}

Let
$f(\alpha,\beta_1,\beta_2,\beta_3) = \frac{\exp( \alpha p_j + x_j' \beta + \xi_j)}{\sum_{k \in \mathcal{F}_t}
  \exp( \alpha p_k + x_k' \beta + \xi_k)}$

Our expectation can then be written as:

\begin{align*}
  \exV{\Pr( i \rightarrow j)} = \int \int \int \int &f(\alpha,\beta) \frac{1}{(2 \pi)^2
                      \sqrt{\theta_{21}\theta_{22}\theta_{23}\theta_{24}}} \exp\left( -
                      \frac{(\alpha-\theta_{11})^2}{2 \theta_{21}^2} \right) \exp\left( -
                      \frac{(\beta_1-\theta_{12})^2}{2 \theta_{22}^2} \right)\\
  &  \exp\left( -
                      \frac{(\beta_2-\theta_{13})^2}{2 \theta_{23}^2} \right) \exp\left( -
                      \frac{(\beta_3-\theta_{14})^2}{2 \theta_{24}^2} \right) d\alpha
                      d\beta_1 d\beta_2 d\beta_3
\end{align*}

We make the following substitutions:
\begin{align*}
  x_1 = \frac{\alpha-\theta_{11}}{\sqrt{2}\theta_{21}} \quad 
  x_2 = \frac{\beta_1-\theta_{12}}{\sqrt{2}\theta_{22}} \quad 
  x_3 = \frac{\beta_2-\theta_{13}}{\sqrt{2}\theta_{23}}\quad
  x_4 = \frac{\beta_3-\theta_{14}}{\sqrt{2}\theta_{24}}  
\end{align*}
Our integral has now become:

\begin{align*}
  \exV{\Pr( i \rightarrow j)} = \frac{1}{\pi^2} \int \int \int \int &\exp(-x_1^2) \exp(-x_2^2)
                                              \exp(-x_3^2)\exp(-x_4^2)\\
  &f( \theta_{11} + \sqrt{2}\theta_{21}x_1, \theta_{12} +
                      \sqrt{2}\theta_{22}x_2, \theta_{13} + \sqrt{2}\theta_{23}x_3, \theta_{14} +
                      \sqrt{2}\theta_{24}x_4) dx_1 dx_2 dx_3 dx_4\\
                    % &= \frac{1}{\pi^2} \int \exp(-x_1^2) \int \exp( -x_2^2) \int \exp( -x_3^2) \int
                    %   \exp( -x_4^2 ) f( \theta_{11} + \sqrt{2}\theta_{21}x_1, \theta_{12} +
                    %   \sqrt{2}\theta_{22}x_2, \theta_{13} + \sqrt{2}\theta_{23}x_3, \theta_{14} +
                    %   \sqrt{2}\theta_{24}x_4) dx_1 dx_2 dx_3 dx_4\\
\end{align*}

We can now apply Gauss-Hermite Quadrature to the set of integrals,
as they are all in the form of $\int_{-\infty}^{\infty} \exp( -x^2) f(x)dx$

\begin{equation*}
  \exV{\Pr( i \rightarrow j)} \approx \frac{1}{\pi^2} \sum_{i=1}^{n_i} \sum_{j=1}^{n_j}
  \sum_{k=1}^{n_k} \sum_{\ell=1}^{n_{\ell}} w_i w_j w_k w_{\ell} f( \theta_{11} + \sqrt{2}\theta_{21}x_i, \theta_{12} +
  \sqrt{2}\theta_{22}x_j, \theta_{13} + \sqrt{2}\theta_{23}x_k, \theta_{14} +
  \sqrt{2}\theta_{24}x_\ell)
\end{equation*}

This is a deterministic function that can be calculated relatively
quickly for a small set of quadrature points. 


\subsection{MPEC Estimation}

Rather than estimate using the Nested Fixed Point algorithm,
mathematical programming under equality constraints will be
employed. At its core, this is simply an estimation procedure where we
let the market shares be equal to the expected demand, and estimate
using the orthogonality condition between $\xi$ and the set of
instruments $z$.

\begin{align*}
  \min_{\alpha,\beta,\xi,\bm{\xi}} \quad & \sum_{t=1}^T \sum_{j=1}^N \bm{\xi}_{jt} \inv{W} \bm{\xi}_{jt}\\
  \text{ subject to: } & \bm{\xi}_{jt} = \xi_{jt} z_{jt}\\
                        & \frac{1}{\pi^2} \sum_q \sum_w \sum_e \sum_r w_q w_w w_e w_r\\
  & \quad \quad D( j, t, (\theta_{11} +
    \sqrt{2}\theta_{21}x_q),(\theta_{12} + \sqrt{2}\theta_{22}x_q),(\theta_{13} +
    \sqrt{2}\theta_{23}x_q),(\theta_{14} + \sqrt{2}\theta_{24}x_q)) = s_{jt}
\end{align*}

Where $D$ is a function such that: 
\begin{equation*}
  D( j, t, \alpha, \beta_1, \beta_2, \beta_3) =  \frac{\exp(\alpha
    p_{jt} + x_{jt} \beta + \xi_{jt})}{\sum_{k \in
      \mathcal{F}_t} \exp(\alpha p_{kt} + x_{kt}\beta + \xi_{kt})}
\end{equation*}

However, this can be simplified rather easily. Consider a set of
parameters in the model $\delta_{jt}$. Let $\delta_{jt}$ be such that:
\begin{equation*}
  \xi_{jt} = \delta_{jt}  - \theta_{11} p_{jt} - \theta_{12} - \theta_{13} x_{jt2} - \theta_{14} x_{jt3}
\end{equation*}

That is, it captures the linear aspects of the normal random variables
in the integration. This reduces the number of variables in the
non-linear constraint from $8$ to $4$, and adds new linear constraints
to account for this. Since solvers are much better at handling linear
constraints, this simplifies matters.

Another simplification arises from altering the constraint on $\bm{w}$.
\begin{equation*}
  \bm{\xi} = \sum_{t=1}^T \sum_{j=1}^N \xi_{jt} z_{jt}
\end{equation*}

This simplifies the number of needed constraints on $\bm{\xi}$ and
reduces the objective function as well.


The optimization question can
now be written as:

\begin{align*}
  \min_{\alpha,\beta,\xi,\bm{\xi},\delta} \quad & \bm{\xi}' \inv{W} \bm{\xi}\\
  \text{subject to: } & \bm{\xi} = \sum_{t=1}^T \sum_{j=1}^N \xi_{jt} z_{jt}\\
  & \frac{1}{\pi^2} \sum_q \sum_w \sum_e \sum_r w_q w_w w_e w_r D( j, t, (
    \sqrt{2}\theta_{21}x_q),(\sqrt{2}\theta_{22}x_q),(\sqrt{2}\theta_{23}x_q),(\sqrt{2}\theta_{24}x_q))
    = s_{jt}\\
  & \xi_{jt} = \delta_{jt} \theta_{11} p_{jt} - \theta_{12} - \theta_{13} x_{jt2} - \theta_{14} x_{jt3}
\end{align*}

Where $D$ is a function such that: 
\begin{equation*}
  D( j, t, \alpha, \beta_1, \beta_2, \beta_3) =  \frac{\exp(\alpha
    p_{jt} + x_{jt} \beta + \xi_{jt})}{\sum_{k \in
      \mathcal{F}_t} \exp(\alpha p_{kt} + x_{kt}\beta + \xi_{kt})}
\end{equation*}

$x,w$ are given by the appropriate Gauss-Hermite samples and
weights, respectively.

We choose $W = Z'Z$ as per suggestion in the problem set. 

This approach was implemented in AMPL using the solver Knitro. The
many different initialization points were selected by Knitro to
approximate the global optimum.

\begin{verbatim}
Final Statistics
----------------
Final objective value               =   1.08291302532658e+02
Final feasibility error (abs / rel) =   1.30e-06 / 2.27e-07
Final optimality error  (abs / rel) =   1.24e-04 / 2.50e-07
# of iterations                     =         12 
# of CG iterations                  =          4 
# of function evaluations           =         14
# of gradient evaluations           =         13
# of Hessian evaluations            =         12
Total program time (secs)           =     297.02255 (   297.006 CPU time)
Time spent in evaluations (secs)    =     290.77496

===============================================================================

Knitro 10.2.0: Locally optimal or satisfactory solution.
objective 108.2913025; feasibility error 1.3e-06
12 iterations; 14 function evaluations

suffix feaserror OUT;
suffix opterror OUT;
suffix numfcevals OUT;
suffix numiters OUT;

thetaOne [*] :=
1  -7.22334
2  -3.02503
3  -5.4718e-05
4  -0.083034


thetaTwo [*] :=
1  0.00605334
2  0.000258687
3  0.0783632
4  0.000432742
\end{verbatim}


\end{document}
