\documentclass[12pt]{article}

\usepackage[margin=1in]{geometry}

\begin{document}
\section{Quantifying Social Influence in an Online Cultural Market}

Quantifying Social Influence in an Online Cultural Market\par
Coco Krumme, Manuel Cebrian, Galen Pickard, Sandy Pentland\par
May 9, 2012\par

\subsection{Abstract}
We revisit experimental data from an online cultural market
in which 14,000 users interact to download songs, and develop a simple
model that can explain seemingly complex outcomes. Our results suggest
that individual behavior is characterized by a two-step process–the
decision to sample and the decision to download a song.  Contrary to
conventional wisdom, social influence is material to the first step
only. The model also identifies the role of placement in mediating
social signals, and suggests that in this market with anonymous
feedback cues, social influence serves an informational rather than
normative role

\subsection{Notes}

This paper somewhat creates a stuctural model, creating this two-stage
discrete choice situation, and focusing on conditional probabilities
between the two stages to predict if a consumer will listen. It still
relies on simulation, and has no endogenized notions of inequality or
unpredictability. The structure only serves to create a predictive
model for whether someone will listen rather than a casual one.

\section{Of songs and men: a model for multiple choice with herding}
Of songs and men: a model for multiple choice with herding\par
Christian Borghesi Jean-Philippe Bouchaud\par
03 March 2007\par

\subsection{Abstract}
We propose a generic model for multiple choice situations in the
presence of herding and compare it with recent empirical results from
a Web-based music market experiment. The model predicts a phase
transition between a weak imitation phase and a strong imitation,
‘fashion’ phase, where choices are driven by peer pressure and
the ranking of individual preferences is strongly distorted at the
aggregate level. The model can be calibrated to reproduce the main
experimental results of Salganik et al. (Science, 311, 854–856
(2006)); we show in particular that the value of the social influence
parameter can be estimated from the data. In one of the experimental
situation, this value is found to be close to the critical value of
the model

\subsection{Notes}

This paper proposes a model for multiple choice, and then simulates
the model with some estimated values used in the paper, and shows that
these simulated results are consistent with the results noted in the
experiment. It is not fitting a model structurally, but does have
quite a bit of structure for their model.


\section{An empirical study of observational learning}
An empirical study of observational learning\par
Peter W. Newberry\par

\subsection{Abstract}
This article provides an empirical examination of observational
learning. Using data from an online market for music, I find that
observational learning benefits consumers, producers of high- quality
music, and the online platform. I also study the role of pricing as a
friction to the learning process by comparing outcomes under
demand-based pricing to counterfactual pricing schemes.  I find that
employing a fixed price (the industry standard) can hamper learning by
reducing the incentive to experiment, resulting in less consumer
surplus, but more expected revenue for the platform

\subsection{Notes}

This paper Structurally estimates the online market for music, but
uses a seperate data set rather than the one used by Salganik.  While
he references Salganik's paper he states that their results show the
long-run effect of what his paper is describing.  'In a project
closely related to the current study, Salganik, Dodds, and Watts
(2006) show descriptive evidence of the long-run effect of social
learning in an experimental music market. I quantify these effects by
estimating a structural model.'

\section{Parameter Evaluation of a Simple Mean-field Model of Social Interaction}

Parameter Evaluation of a Simple Mean-field Model of Social Interaction\par
Ignacio Gallo, Adriano Barra and Pierluigi Contucci\par

\subsection{Abstract}
The aim of this work is to implement a statistical mechanics theory of
social interaction, generalizing econometric discrete choice models. A
class of simple mean-field discrete models is introduced and discussed
both from the theoretical and phenomenological point of view. We
propose a parameter evaluation procedure and test it by fitting the
model against three families of data coming from different cases: the
estimated interaction parameters are found to have similar positive
values, giving a quantitative confirmation of the peer imitation
behavior found in social psychology. Furthermore, all the values of
the interaction parameters belong to the phase transition regime
suggesting its possible role in the study of social systems

\subsection{Notes}

This is an attempt to generalize the discrete choice model further
with interaction parameters, but not really related to 'learning'
about the enviorment, just enjoying what they believe others to
enjoy. This data is then used in census data on several social matters
in Italy. This would only really be relevant for adding to papers
using the discrete choice model.


\section{Aligning Popularity and Quality in Online Cultural Markets}
Aligning Popularity and Quality in Online Cultural Markets\par

Pascal Van Hentenryck, Andres Abeliuk, Franco Berbeglia, Felipe
Maldonado, Gerardo Berbeglia\par

\subsection{Abstract}
Social influence is ubiquitous in cultural markets and plays
an important role in recommendations for books, songs, and
news articles to name only a few. Yet social influence is often
presented in a bad light, often because it supposedly increases
market unpredictability. Here we study a model of trial-offer
markets, in which participants try products and later decide
whether to purchase. We consider a simple policy which re-
covers product quality and ranks the products by quality when
presenting them to market participants. We show that, in this
setting, market efficiency always benefits from social influ-
ence. Moreover, we prove that the market converges almost
surely to a monopoly for the product of highest quality, mak-
ing the market both predictable and asymptotically optimal.
Computational experiments confirm that the quality ranking
policy quickly identifies ``blockbusters'', outperforms other
policies, and is highly predictable

\subsection{Notes}

This paper finds results opposite to Salganik's paper, directly
contrasting the results from his experimental approach. He uses
simulations to try to model the music lab experiments. His estimate of
quality is derived from the data used in the Music Lab even if there
is not a structural estimation from the data set. This paper finds
instead that markets are still efficient, and converges to a monopoly.

\section{The success of art galleries: a dynamic model with competition and information effects
}
The success of art galleries: a dynamic model with competition and
information effects\par


Aloys Prinz, Jan Piening,Thomas Ehrmann\par

08 April 2014\par

\subsection{Abstract}
An intrinsic characteristic of cultural goods is the unpredictability
of their economic success. Arts goods in particular share
characteristics with credence, inspection, and experience
goods. Accordingly, art collectors rely on the experience and the
reputation of art galleries when investing in artwork. Some
qualitative sociological studies have found that only a few very
successful galleries represent the bulk of the most visible and most
successful artists (e.g., Crane in The transformation of the
avant-garde: the New York art world, 1940–1985. The University of
Chicago Press, Chicago, 1989; Currid in The Warhol economy: How
fashion, art and music drive New York City. Princeton University
Press, Princeton, 2007). This paper investigates the success of art
galleries in a dynamic model, which elaborates different statistical
processes that allow us to analyze the development of different types
of success distributions in the market for art galleries. Instead of
applying standard economic analysis only, we employ methods from
statistical physics to construct a model of gallery investment and
competition. Our model entails information, competition, and
innovation effects. Subsequently, art market data are used to test
which version of the model fits best. We find that the lognormal
distribution provides the best fit and conclude that the data
generating process is compatible with the version of the model, which
entails an inhomogeneous geometric Brownian motion. Hence, the success
of art galleries depends strongly on information and innovation
effects, but is hardly affected by competition effects. We argue that
the superstar effect in the case of art galleries can be understood as
an appropriation of search and entrance costs, which emerge whenever
consumption requires special knowledge and social inclusion

\subsection{Notes}

This paper examines art-galleries rather than music markets, and
treats them as a two-sided market. They estimate the dynamics of the
world with a pretty nasty mean-reverting brownian motion, then fit
art-market data. Their conclusion is that the information provided
reduces search costs for individuals and that this drives the
super-star effect.


\section{Charts and demand: Empirical generalizations on social influence}
Charts and demand: Empirical generalizations on social influence\par

Olaf Maecker, Nadja Sophia Grabenströer, Michel Clement, Mark
Heitmann\par

2 September 2013\par

\subsection{Abstract}
Social influence on consumer behavior has long been a subject of
academic research in various scientific fields.  According to research
by Salganik, Dodds, and Watts (2006), music demand is a function of
social influence between consumers. Market concentration tends to
increase when information on demand becomes publicly available. In
addition, stochastic agglomeration caused by social influence
decreases the predictability of market success. These heavily cited
findings challenge traditional market research and provide important
insights on the impact of social media and sales charts. We test the
stability of their results by replicating the study on music demand in
a slightly different setting. We further investigate the
generalizability of findings by probing other product categories and
different phases of purchase decisions, i.e., interest, consideration,
and actual demand.  Across all categories and across all dependent
variables, we are able to replicate the direction of the effects.  We
do, however, consistently obtain smaller effect sizes than reported in
the original paper

\subsection{Notes}

This is a separate replication experiment of the original paper, and
may be useful for creating a test set on any predictive models
developed. They expanding the experiment to also include movies and
scarves, but only so far as self-reporting on interest. Since this is
not the same as downloading a song, I do not know whether that
particular data could be useful.


\section{Social Networks and the Diffusion of User-Generated}
Social Networks and the Diffusion of User-Generated
Content: Evidence from YouTube\par
Anjana Susarla, Jeong-Ha Oh, Yong Tan\par
March 2012\par

\subsection{Abstract}
This paper is motivated by the success of YouTube, which is attractive
to content creators as well as corporations for its potential to
rapidly disseminate digital content. The networked structure of
interactions on YouTube and the tremendous variation in the success of
videos posted online lends itself to an inquiry of the role of social
influence. Using a unique data set of video information and user
information collected from YouTube, we find that social interactions
are influential not only in determining which videos become successful
but also on the magnitude of that impact. We also find evidence for a
number of mechanisms by which social influence is transmitted, such as
(i) a preference for conformity and homophily and (ii) the role of
social networks in guiding opinion formation and directing product
search and discovery. Econometrically, the problem in identifying
social influence is that individuals’ choices depend in great part
upon the choices of other individuals, referred to as the reflection
problem. Another problem in identification is to distinguish between
social contagion and user heterogeneity in the diffusion process. Our
results are in sharp contrast to earlier models of diffusion, such as
the Bass model, that do not distinguish between different social
processes that are responsible for the process of diffusion. Our
results are robust to potential self-selection according to user
tastes, temporal heterogeneity and the reflection
problem. Implications for researchers and managers are discussed

\subsection{Notes}

This paper is examining social influence from a very
network-orientated approach. It examines youtube data rather than
music data. It applies a structural approach, and attempts to control
for many interaction terms and endogeneity concerns.



\end{document}
