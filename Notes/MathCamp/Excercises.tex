\documentclass[12pt, letterpaper]{paper}
\usepackage[utf8]{inputenc}
\usepackage[T1]{fontenc}
\usepackage{graphicx}
\usepackage{grffile}
\usepackage{longtable}
\usepackage{wrapfig}
\usepackage{rotating}
\usepackage[normalem]{ulem}
\usepackage{textcomp}
\usepackage{amssymb}
\usepackage{hyperref}
\usepackage{Schwieg}


\author{Timothy Schwieg}
\date{\today}
\title{Math Camp Exercises}


\begin{document}

\maketitle

\section{Basic Topology and Linear Algebra}


\begin{question}
  Show that the \emph{Kullback-Leibler distance} is not a valid metric

  \begin{proof}
    Note that a valid metric must have the property that: $\met{f,g} =
    \met{g,f}$. This is not true.

    Let $f \sim exp(1), g \sim exp(2)$. 
    \begin{align*}
      \met{f,g} &= \int_{\setR} \log \left ( \frac{ \exp (  -x )}{ 2 \exp ( -2x  )} \right ) \exp (  -x ) dx\\
                &= \int_{\setR} x \exp (  -x ) \log \left ( \frac{1}{2} \right ) dx\\
      \met{g,f} &= \int_{\setR} \log \left ( \frac{2 \exp ( -2x )}{\exp ( -x )}
                  \right ) 2 \exp (  -2x  )\\
                &= \int_{\setR} \frac{2 \exp ( -2x )}{x} \log ( 2 ) dx\\      
    \end{align*}

    Clearly those integrands, and therefore the integrals are not equal.
  \end{proof}
\end{question}

\begin{question}
  \begin{enumerate}
  \item Give an example of a set that is both open and closed
  \item Give an example of a set that is neither open nor closed
  \end{enumerate}

  \begin{proof}
    \begin{enumerate}
    \item $\setR$ is both open and closed, as all convergent sequences
      in $\setR$ have their limits in $\setR$ by definition of being
      convergent, and clearly any neighborhood in $\setR$ is a subset
      of $\setR$.
    \item Consider the set $E = \set{ \frac{1}{n} | n \in \setN}$. We may
      first note that this set has a limit point of $0$, but $0 \notin
      E$. Therefore $E$ is not closed.

      Consider any neighborhood of any point in $E$. This neighborhood
      contains a point not contained in $E$. As $E$ is countable and
      any neighborhood is uncountable. Therefore it cannot be that the
      neighborhood is a subset of $E$.
    \end{enumerate}
  \end{proof}
\end{question}

\begin{question}
  Give an example to show that a set can be open in one metric space
  but not open in another.

  \begin{proof}
    Consider the segment $(a,b)$. This segment is clearly open in
    $\setR$, however, if the embedded set is $\setR^2$ it is not open,
    as any neighborhood would contain points with $y$ coordinates
    that are not equal to zero, and therefore is not a subset of $(a,b)$.
  \end{proof}
\end{question}

\begin{question}
  Show that $C([0,1])$ is complete.

  \begin{proof}
    Consider any Cauchy sequence $\seq{f_n}$ in $C([0,1])$. For some
    $\epsilon > 0$, $\met{f_m, f_n} < \epsilon \quad \forall m,n \geq N(\epsilon)$. For any $x_0 \in
    [0,1]$
    \begin{equation*}
      \epsilon > \sup_{x \in [0,1]} \abs{ f_m(x) - f_n(x)} > \abs{ f_m(x_0) -
        f_n (x_0)}
    \end{equation*}


    Note that $f_n(x_0)$ is a sequence in $\setR$ instead of
    $C([0,1])$. Since $\setR$ is complete, this implies that
    $f_n(x_0)$ converges to a point in $\setR$. Collect these limits
    into a function $f(x_0)$.

    We must now show that $f(x)$ is continuous. Consider any point
    $x_0 \in [0,1]$. Choose
    $n \geq \ceil{ \sup_{x \in [0,1]} N( \frac{\epsilon}{3}, x)}$. That is, let
    $N$ be the supremum of $N$ taken for all the convergent sequences
    $\seq{f_n(x)}$ when $\epsilon= \frac{\epsilon}{3}.$ Applying the definition of
    continuity of $f_n(x)$ at $x_0$ for $\frac{\epsilon}{3}$, for any
    $x$ such that: $\met{x,x_0} < \delta( \frac{\epsilon}{3}, x_0)$.
    \begin{align*}
     \abs{ f(x) - f(x_0)} &= \abs{f(x) - f(x_0) + f_n(x) - f_n(x) +
                            f_n(x_0) -f_n(x_0)}\\
                          &= \abs{ \brak{ f_n(x_0) - f(x_0)} - \brak{f_n(x_0) -
                              f_n(x)} + \brak{f(x) - f_n(x)}}\\
                          &\leq \abs{ f_n(x_0) - f(x_0)} + \abs{ f_n(x_0)
                            - f_n(x)} + \abs{ f(x) - f_n(x)}\\
                          &< \frac{\epsilon}{3} + \frac{\epsilon}{3} + \frac{\epsilon}{3}\\
                          &< \epsilon\\
    \end{align*}

    As $[0,1]$ is compact, and $f$ is continuous, it is compact
    valued, and therefore the supremum is obtained.
    \begin{equation*}
      \sup_{x \in [0,1]}\abs{ f_n(x) - f(x)}  = \max_{x \in [0,1]} \abs{
        f_n(x) - f(x)} < \epsilon
    \end{equation*}
  \end{proof}
\end{question}

\begin{question}
  Let $(X,\met{.})$ be a metric space, show that the collection of
  open sets in $X$ is a topology.

  \begin{proof}
    \begin{enumerate}
    \item The empty set is an open set vacuously.

      Consider any point $x_0$ in $X$. A neighborhood of $x_0$
      consists of points $x \in X$ such that $\met{x,x_0} < \delta$. Clearly
      this is a subset of $X$ and $X$ is therefore open.
    \item Consider a collection $\seq{O_k}_{k=1}^n \in X$. Take any
      point $x \in \cap_{k=1}^n O_k$. As each $O_k$ is open, there exists
      $\delta_k$ that defines a neighborhood of $x$ such that $N_{\delta_k} \subset
      O_k$. Taking $\delta = \min \{ \delta_k\}$ we have a neighborhood with a
      strictly positive radius, as it is the minimum of a finite set
      of real numbers. All the points in $N_{\delta}(x) \subset \cap_{k=1}^n O_k$ and
      therefore that set is open.
    \item Consider a cover $\mathcal{F}$, Take any point $x \in \cup_{O \in
        \mathcal{F}} O$. $x$ belongs to at least one of the sets $E \in
      \mathcal{F}$. As this set is open, there is a neighborhood of
      $x$ that is a subset of $E$. As this neighborhood is contained
      in E, it is contained in $\cup_{O \in \mathcal{F}} O$ and therefore
      $\cup_{O \in \mathcal{F}} O$ is open.
    \end{enumerate}
  \end{proof}
\end{question}

\begin{question}
  Let $\succeq$ be a preference relation on $\setR_+^n$ that is complete and
  transitive. Show that: $\{ (x,y) \in \setR_+^n x \setR_+^n | x \succeq y \}$
  is closed is equivalent to $\succeq$ is continuous.

  \begin{proof}
    Assume that $\succeq$ is continuous. Let $A = \{ (x,y) \in \setR_+^{2n} |
    x \succeq y\}$. Then $\compl{A} = \{ (x,y) \in \setR_+^{2n} | y \succ x
    \}$. Note that $\compl{A} = \cup_{x \in \setR_+^n} ( x, \compl{\succeq(x)} )$. Take a
    point in $\compl{A}$, $(x,y)$. 

    Since $y \in \compl{\succeq(x)}$, and that set is open, there is an
    uncountable number of other $y'$ such that: $y' \succ x$. Fix such a
    $y'$. Note that $y' \succ x$. Note that $x$ is in the complement of
    the lower contour set of $y'$.  Which is an open set, so there
    must be a neighborhood of points where $\met{x', x} < \delta_x$ and
    $y' \succ x'$.

    
  \end{proof}
\end{question}

\begin{question}
  Prove:
  \begin{enumerate}
  \item $f$ is continuous at $x_0 \in X$
  \item For any $\seq{x_n}$ such that $\seq{x_n} \to x_0$, $\lim f(x_n)
    = f(x_0)$.
  \item For any $\epsilon > 0$, $\exists \delta > 0$ such that $\met{x,y} < \delta \Rightarrow
    \met{f(x), f(y)} < \epsilon$.
  \end{enumerate}
  \begin{proof}
    \begin{itemize}
    \item $(1) \Rightarrow (3)$: Let $f$ be continuous at $x_0$, Fix
      $\epsilon > 0$, take the neighborhood of $f(x_0)$ with radius
      $\epsilon$. Since this set is open, the pre-image of $N_{\epsilon}(f(x_0))$,
      $\inv{f}( N_{\epsilon}(f({x_0})))$ is an open set. Therefore it
      contains a neighborhood as a subset of itself. Let $\delta$ be the
      radius of this neighborhood. Every point in that neighborhood
      satisfies $\met{x_0,y} < \delta$ and $\met{f(x_0),f(y)} < \epsilon$.
      
    \item $(3)\Rightarrow(2)$. Fix $\epsilon > 0$, apply $(3)$ to obtain a
      $\delta > 0$. Consider any sequence $\seq{x_n} \to x_0$. Apply the
      definition of convergence of a sequence to obtain $N(\delta)$ such that
      $\forall n \geq N, \met{x_n, x_0} < \delta$. By $(3)$, for each of these
      $x_n, \met{f(x_n), f(x_0)} < \epsilon$. Thus this $N(\delta)$ is exactly the
      $N$ for $\seq{f(x_n)}$ to converge, and $(2)$ is satisfied.
      
    \item $(2)\Rightarrow(3)$. Approach by contrapositive.  Assume that
      $(3)$ is false, then there exists some $\epsilon > 0$ such that
      $\forall \delta > 0, \exists x \in E$ where
      $\met{x,x_0} < \delta$ but $\met{f(x), f(x_0)} \geq \epsilon$. This is true for
      all $\delta$. So for $\delta_n = \frac{1}{n}$, apply this rule at each
      $n$ to obtain an $x_n$. Clearly $x_n \to x_0$, but since
      $\met{f(x_n),f(x_0)} > \epsilon \quad \forall n$, the sequence
      $f(x_n)$ cannot converge to $f(x_0)$.
    \item $(3)\Rightarrow(1)$. Consider any open set $V$. Take $x_0$ such that
      $f(x_0) \in V$. As $V$ is open, $\exists \epsilon > 0$ such that $\met{f(x_0),
        v } < \epsilon \Rightarrow v \in V$. Using $(3)$, $\exists\delta > 0$ such that
      $\met{f(x), f(x_0)} < \epsilon$ if $\met{x,x_0} < \delta$. This implies that
      $x \in \inv{f}(V)$. This is true for all $x$ such that
      $\met{x,x_0} < \delta$ and thus there is a neighborhood of $x_0$ that
      is a subset of $\inv{f}(V)$.
    \end{itemize}
  \end{proof}
\end{question}

\begin{question}
  Consider a Betrand competition game. That is, there are two firms,
  both of them are facing a market with perfectly inelastic demand
  and firm $i$ has a marginal cost of production $c_i$. Suppose that
  firms are competing by setting lower prices. That is, if $p_i <
  p_j$ then firm $i$ wins the whole market and gets profit $(p_i -
  c_i)$ and firm $j$ loses and gets zero. Suppose also that whenever
  there is a tie, each gets half the market. Show:
  \begin{enumerate}
  \item When $c_1 = c_2 = c > 0$ there exists a unique pure strategy
    Nash equilibrium under which $p_1 = p_2 = c$.
    
  \item When $0 < c_1 < c_2$ there is no pure strategy
    equilibrium. What went wrong? Is there any assumption you can make
    to get around this?
  \end{enumerate}
  \begin{proof}
    \begin{enumerate}
    \item Approach by cases:
      \begin{itemize}
      \item Case: $p_1$ or  $p_2 < c$. At least one firm is making a loss, and
        have an incentive to raise their prices to at least $c$.
      \item Case: $p_1 = p_2 > c$. Either firm has an incentive to
        reduce their price by $\epsilon > 0$ and obtain the entire market
        while still making profit. So this cannot be a Nash
        Equilibrium.
      \item $p_1, p_2 \geq c, p_1 \neq p_2$ The firm with the lower price
        wishes to raise its price to some number between $p_1, p_2$ in
        order to increase its profit. This cannot be a Nash
        equilibrium then.
      \item $p_1 = p_2 = c$. Both firms are indifferent between
        raising their prices and earning zero profit and maintaining
        prices and earning zero profit. Both are strictly averse to
        lowering their prices as they would then earn negative
        profit. This means that both prices are a best response to the
        other prices, and this is a Nash Equilibrium.
      \end{itemize}
    \item When $0 < c_1 < c_2$, Firm 1 now wishes to lower his price
      from $c_2$, and Firm 2 has no desire to drop their price below
      $c_2$ as they would earn negative profit. However Firm 1 would
      like to have as high of a profit as possible, but when $p_1 =
      p_2$ they will earn less profit than $p_1 = p_2 - \epsilon$. This means
      that there is no maximum to their profit function, and they can
      always raise their price and earn more profit, but must keep
      their price below $c_2$. 

      One assumption to avoid this pitfall is to discretize the price
      space. If the prices are forced to be in whole cents, then Firm
      1 will wish to charge $c_2 - .01$ and this will be the maximum
      profit that they are capable of earning.
    \end{enumerate}
  \end{proof}
\end{question} 

\begin{question}
  Let $X$ be a compact topological space. Suppose that $W : C(X) \to
  C(X)$ is a self map on $C(X)$, the collection of continuous
  functions under the norm $\norm{f}_{\infty} := \max_{x \in X} \abs{
    f(x)}$. Show that $W$ is a contraction if:
  \begin{itemize}
  \item $W$ is increasing. That is, for any $f,g \in X, f \geqq g, W(f) \geq
    W(g)$.
  \item For any $f \in X, \alpha \in \setR_+$: $W(f+\alpha) \leq W(f) + \beta \alpha$ for some
    $\beta \in (0,1).$ Furthermore, use this to show that the Bellman
    operator $(3)$ is a contraction.
  \end{itemize}
  \begin{proof}
    We wish to show that $\norm{W(f)-W(g)} \leq r \norm{ f,g}$ where $r
    \in [0,1)$. 
  \end{proof}
\end{question}


\section{Integration and Differentiation}



\end{document}